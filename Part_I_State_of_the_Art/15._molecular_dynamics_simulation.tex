
\chapter{Molecular Dynamics \& Monte Carlo Simulations\label{chpt:reference-method}}

As reference method {[}Allen and Tildesley{]} {[}Frenkel and Smit{]}

The data we used in this thesis issue from \acs{MD} and \acs{MC}
simulations are in two aspects: (1) The \acs{DCF} of bulk water produced
by the work of Zhao et al. {[}ref{]} and Belloni et al. {[}ref{]}.
(2) The \acf{RDF} of ions and molecules. The first two topics give
a brief review of the principe of \acs{MD} and \acs{MC}, the last
two detail the generation of data for the first two.


\section{Molecular Dynamics}


\subsection{Principle}


\subsection{Determination of free energy: umbrella sampling}


\section{Monte Carlo Method}


\section{Direct correlation function of water}

described the direct correlation functions (\acs{DCF}) used during
the thesis, including different sources and forms, as well as their
comparison.


\subsection{Dipole DCF from molecular dynamics simulation}


\subsection{Extraction of DCF from bulk Monte Carlo simulation}

This DCF set is calculated by Belloni \textit{et al.} \citep{Luc_2012}
and is presented here briefly for the purpose of clarification. First
rotational invariant of the Fourier transform of the total correlation
function h

example:

\[
\hat{c}^{000}(k)=\frac{\hat{h}^{000}(k)}{1+n_{0}\hat{h}^{000}(k)}
\]


$g$ accumulated, solve the inverse OZ equation to find $c$

norm of wave vector $\left\Vert \mathbf{q}\right\Vert =\left\{ i\cdot\Delta_{q}\mid0\leq i\leq n,i\in\mathbb{Z}\right\} $,
beginning from the 10th line, where $\Delta_{q}$ is the minimum difference
between two norm $q$. 


\subsection{Conventions}

whereas I take the notation of Wertheim and Hansen, or in k-space

\begin{align}
\Phi^{000} & =1\nonumber \\
\Phi^{011} & =\hat{\mathbf{k}}\cdot\mathbf{\Omega}_{1}\nonumber \\
\Phi^{101} & =\hat{\mathbf{k}}\cdot\mathbf{\Omega}_{2}\nonumber \\
\Phi^{110} & =\mathbf{\Omega}_{1}\cdot\mathbf{\Omega}_{2}\\
\Phi^{112} & =3\mathbf{(\hat{\mathbf{k}}\cdot\mathbf{\Omega}_{1})(\hat{\mathbf{k}}\cdot\mathbf{\Omega}_{2})-\Omega}_{1}\cdot\mathbf{\Omega}_{2}\nonumber 
\end{align}
Different rotational invariant projections from Luc’s c. Luc defines
(ac- cording to Blum)

\begin{align}
\Phi^{000} & =1\nonumber \\
\Phi^{011} & =i\mathbf{k}\cdot\mathbf{\Omega}_{1}=i\cos\theta_{1}\nonumber \\
\Phi^{101} & =i\mathbf{k}\cdot\mathbf{\Omega}_{2}=i\cos\theta_{2}\nonumber \\
\Phi^{110} & =-\sqrt{3}\mathbf{\Omega}_{1}\cdot\mathbf{\Omega}_{2}=-\sqrt{3}(\sin\theta_{1}\sin\theta_{2}\cos\phi_{12}+\cos\theta_{1}\cos\theta_{2})\\
\Phi^{112} & =\sqrt{\frac{3}{10}}\left[3\mathbf{(\mathbf{k}\cdot\mathbf{\Omega}_{1})(\mathbf{k}\cdot\mathbf{\Omega}_{2})-\Omega}_{1}\cdot\mathbf{\Omega}_{2}\right]=\sqrt{\frac{3}{10}}\left(2\cos\theta_{1}\cos\theta_{2}-\sin\theta_{1}\sin\theta_{2}\cos\phi_{12}\right)\nonumber 
\end{align}



\subsection{Comparison with non-coupling dipole DCF in MDFT ($n_{\max}=1$)\label{sub:Comparison-with-non-coupling}}

$c_{s}^{000},c_{\Delta}^{110},c_{d}^{112}$ 


\subsection{Comparison with respect to $n_{\max}$}

first rotational invariants $m{}_{\mathrm{max}}=1$ (4 independent
projections $c_{s}^{000},c_{\Delta}^{110},c_{d}^{112}$ and $c_{+}^{011}=-c_{-}^{101}$).


\subsection{Transform between ck\_angular and projections}

$\hat{c_{q}}(\cos\theta_{1},\cos\theta_{2},\phi,\psi_{1},\psi_{2})$

olumns: $\cos\theta_{1}$, $\cos\theta_{2}$, $\psi_{1}$, $\psi_{2}$,
$\phi=\phi_{1}-\phi_{2}$, $\mathrm{Re}(\hat{c})$, $\mathrm{Im}(\hat{c})$;
with $(\mathrm{n}_{\theta}+1)\mathrm{n}_{\theta}\mathrm{n}_{\phi}\mathrm{n}_{\psi}^{2}/4$
items as only $\cos\theta_{1}+\cos\theta_{2}>0$ and $\phi\in\left[0,\pi\right]$
is stocked, where $n_{\mathrm{(angle)}}$ is the number of angles
for each dimension. 


\section{Umbrella sampling}
