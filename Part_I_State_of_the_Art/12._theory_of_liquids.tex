
\chapter{Statistical Mechanics of Atomic Fluids\label{chpt:statistical-mechanics}}

Statistical mechanics serves to deduce thermodynamic quantities from
the Hamiltonian of any given system. In this section, we present some
basic formalism for a classical atom-like spherical solvent model
in grand canonical ensemble ($\mu$,$V$,$T$). Firstly, we introduce
the relations between the statistical mechanics and thermodynamic
quantities. Then we change the view to the structure of the solvent.
The two theories we use in this thesis, here referred to as \acs{IET}
and c\acs{DFT}, as well as their equivalency, are presented with
brief derivations in the following sections. The majority of these
sections are based on the book by Hansen \& McDonald \citep{HANSEN_2ed,Hensen-McDonald},
and the articles and notes of Evans \citep{Evans_1979,Evans_1984,Evans_1992}.
A very detailed review is done by Wu \textit{et al.} \citep{Wu2009}
to the same purpose, thus here we only introduce the concepts that
will be useful to understand this thesis.

\section{Hamiltonian and ensemble properties}

Once we define a spherical solvent model, of which the movement only
depends on its position and momentum $(\mathbf{r},\mathbf{p})$, the
instantaneous state (phase point, micro-state) of an $N$-particle
solvent system is specified by $3N$ coordinates $\mathbf{r}^{N}\equiv\mathbf{r}_{1},\ldots,\mathbf{r}_{N}$
and $3N$ momenta $\mathbf{p}^{N}\equiv\mathbf{p}_{1},\ldots,\mathbf{p}_{N}$.
The internal energy of particles in a system is characterized by its
Hamiltonian:
\begin{equation}
H_{N}(\mathbf{r}^{N},\mathbf{p}^{N})=K_{N}(\mathbf{p}^{N})+V_{N}(\mathbf{r}^{N})+V_{N}^{\mathrm{ext}}(\mathbf{r}^{N})
\end{equation}
where

\begin{tabular*}{1\columnwidth}{@{\extracolsep{\fill}}l>{\raggedright}p{0.9\columnwidth}}
$K_{N}(\mathbf{p}^{N})$ & $={\displaystyle \sum_{i=1}^{N}\frac{\mathbf{p}_{i}^{2}}{2m}}$ is
the kinetic energy;\tabularnewline
$V_{N}(\mathbf{r}^{N})$ & $={\displaystyle \sum_{i<j}^{N}u(\left|\mathbf{r}_{i}-\mathbf{r}_{j}\right|)+3\,\mathrm{body}+\ldots}$
is the interatomic potential energy $\mathcal{U}(\mathbf{r}^{N})$;\tabularnewline
$V_{N}^{\mathrm{ext}}(\mathbf{r}^{N})$ & $={\displaystyle \sum_{i=1}^{N}}V_{\mathrm{ext}}(\mathbf{r}_{i})$
is the potential energy arising from the interaction of the particles
with the external field (e.g. a solute).\tabularnewline
 & \tabularnewline
\end{tabular*}

The grand potential, characteristic thermodynamic state function for
the grand canonical ensemble, which depends on the chemical potential
$\mu$, the volume $V$ and the temperature $T$, is linked with the
statistical mechanics quantities with the relation:
\begin{equation}
\varOmega(\mu,V,T)=-k_{\mathrm{B}}T\ln\varXi\label{eq:1.2}
\end{equation}
where
\begin{eqnarray}
\varXi & = & \sum_{N=0}^{\infty}\frac{e^{\beta\mu N}}{h^{3N}N!}\int\mathrm{d}\mathbf{r}^{N}\mathrm{d}\mathbf{p}^{N}e^{-\beta H_{N}(\mathbf{r}^{N},\mathbf{p}^{N})}\\
 & = & \sum_{N=0}^{\infty}\dfrac{\text{1}}{N!}\int\mathrm{d}\mathbf{r}^{N}e^{-\beta V_{N}(\mathbf{r}^{N})}\left(\prod_{i=1}^{N}\frac{e^{\beta V_{\mathrm{int}}(\mathbf{r}_{i})}}{\Lambda^{3}}\right)
\end{eqnarray}
 is the grand partition function, with $\Lambda=\left(2\pi\beta\hbar^{2}/m\right)^{-\frac{1}{2}}$
the de Broglie thermal wavelength, and
\begin{equation}
V_{\mathrm{int}}(\mathbf{r}_{i})=\mu-V_{\mathrm{ext}}(\mathbf{r}_{i})\label{eq:1.5}
\end{equation}
the intrinsic chemical potential. 

We can also define the intrinsic free energy: \marginpar{$N=\int\mathrm{d}\mathbf{r}\bar{n}(\mathbf{r})$ is the number of
particles in canonical ensemble, but the formulae (\ref{eq:1.f-int})
and (\ref{eq:delta-f-int}) are also available for grand canonical
ensemble.}
\begin{eqnarray}
F_{\mathrm{int}} & = & F-\int\mathrm{d}\mathbf{r}\bar{n}(\mathbf{r})V_{\mathrm{ext}}(\mathbf{r})\nonumber \\
 & = & \varOmega+\mu N-\int\mathrm{d}\mathbf{r}\bar{n}(\mathbf{r})V_{\mathrm{ext}}(\mathbf{r})\nonumber \\
 & = & \varOmega+\int\mathrm{d}\mathbf{r}\bar{n}(\mathbf{r})V_{\mathrm{int}}(\mathbf{r})\label{eq:1.f-int}
\end{eqnarray}
where 
\begin{equation}
\bar{n}(\mathbf{r})=\left\langle \varrho(\mathbf{r})\right\rangle =\left\langle \sum_{i=1}^{N}\delta(\mathbf{r}-\mathbf{r}_{1})\right\rangle 
\end{equation}
is the density profile of instantaneous density $\varrho(\mathbf{r})$
distribution at equilibrium.

The differential form of $F_{\mathrm{int}}$ is
\begin{equation}
\delta F_{\mathrm{int}}=-S\delta T+\int\mathrm{d}\mathbf{r}\delta\bar{n}(\mathbf{r})V_{\mathrm{int}}(\mathbf{r})\label{eq:delta-f-int}
\end{equation}
with $S$ the entropy.

The internal energy of the solvent contains two contributions, one
due to the kinetic energy of the particles, $K_{N}(\mathbf{p}^{N})$,
and the other linked to the interaction between particles, $V_{N}(\mathbf{r}^{N})$.
When the fluid is a perfect gas, which means $V_{N}=0$, it can be
easily derived from eq. (\ref{eq:1.2}-\ref{eq:1.5}) that $F_{\mathrm{int}}$
has the following expression:
\begin{equation}
F_{\mathrm{id}}=k_{\mathrm{B}}T\int\mathrm{d}\mathbf{r}\bar{n}(\mathbf{r})\left[\ln\left(\Lambda^{3}\bar{n}(\mathbf{r})\right)-1\right]
\end{equation}
When interactions between particles are accounted for, the total expression
of $F_{\mathrm{int}}$ is:
\begin{equation}
F_{\mathrm{int}}=F_{\mathrm{id}}+F_{\mathrm{exc}}\label{eq:f-int-def}
\end{equation}
and the form of $F_{\mathrm{exc}}$ will be detailed in later sections.

\section{Functional derivatives and distribution functions}

The structure of the solvent in the grand canonical ensemble can be
characterized by its $n$-particle density 
\begin{equation}
\rho^{(n)}(\mathbf{r}^{n})=\dfrac{1}{\varXi}\sum_{N=n}^{\infty}\dfrac{1}{(N-n)!}\int\mathrm{d}\mathbf{r}^{\left(N-n\right)}e^{-\beta V_{N}(\mathbf{r}^{N})}\left(\prod_{i=1}^{N}\frac{e^{\beta V_{\mathrm{int}}(\mathbf{r}_{i})}}{\Lambda^{3}}\right)\label{eq:1.def-rho}
\end{equation}
which means the probability to find $n$ particles in a volume element
$\mathrm{d}\mathbf{r}^{n}$. In particular, the probability to find
one particle in a volume element is the solvent density $\rho^{(1)}(\mathbf{r})=\bar{n}(\mathbf{r})$,
that 
\begin{equation}
\int\rho^{(1)}(\mathbf{r})\mathrm{d}\mathbf{r}=\left\langle N\right\rangle 
\end{equation}
where $\left\langle N\right\rangle $ is the ensemble average of the
number of particles, that is to say the average number of particles
at equilibrium. $\rho^{(n)}(\mathbf{r}^{n})$ becomes $\rho^{n}$
if the system is homogeneous. It can be proven that
\begin{equation}
\dfrac{\delta\varOmega}{\delta V_{\mathrm{int}}(\mathbf{r})}=-\rho^{(1)}(\mathbf{r})
\end{equation}

The corresponding $n$-particle distribution function is defined as:
\begin{equation}
g^{(n)}(\mathbf{r}^{n})=\dfrac{\rho^{(n)}(\mathbf{r}^{n})}{\prod_{i=1}^{n}\rho^{(1)}(\mathbf{r}_{i})}
\end{equation}
such that $g^{(n)}(\mathbf{r}^{n})\rightarrow1$ when all pairs of
particles becomes sufficiently large.

The two-particle pair distribution function (\acs{PDF}), $g^{(2)}(\mathbf{r}_{1},\mathbf{r}_{2})$,
is one of the most important quantities in the theory of liquids.
Its corresponding pair correlation function (\acs{PCF}) is defined
as:
\begin{equation}
h^{(2)}(\mathbf{r}_{1},\mathbf{r}_{2})=g^{(2)}(\mathbf{r}_{1},\mathbf{r}_{2})-1
\end{equation}
which vanishes when $\left|\mathbf{r}_{1}-\mathbf{r}_{2}\right|\rightarrow\infty$.

If we define the density-density correlation function as: \marginpar{For any ensemble.}
\begin{equation}
H^{(2)}(\mathbf{r}_{1},\mathbf{r}_{2})=\rho^{(1)}(\mathbf{r}_{1})\rho^{(1)}(\mathbf{r}_{2})h^{(2)}(\mathbf{r}_{1},\mathbf{r}_{2})+\rho^{(1)}(\mathbf{r}_{1})\delta(\mathbf{r}_{1}-\mathbf{r}_{2})\label{eq:H-definition}
\end{equation}
which means the correlation \citep{Correlation_function_wiki} between
the instantaneous fluctuation of particle density from its ensemble
average, it can be proven that
\begin{equation}
\dfrac{\delta\varOmega^{2}}{\delta V_{\mathrm{int}}(\mathbf{r}_{1})\delta V_{\mathrm{int}}(\mathbf{r}_{2})}=-\beta H^{(2)}(\mathbf{r}_{1},\mathbf{r}_{2})=-\dfrac{\delta\rho^{(1)}(\mathbf{r}_{1})}{\delta V_{\mathrm{int}}(\mathbf{r}_{2})}
\end{equation}

As an analogue, the direct correlation function (\acs{DCF}) is defined
as the derivative of the excess free energy functional $F_{\mathrm{exc}}[\rho]$:

\begin{equation}
c^{(1)}(\mathbf{r})=-\dfrac{\delta(\beta F_{\mathrm{exc}}[\rho^{(1)}])}{\delta\rho^{(1)}(\mathbf{r})}
\end{equation}
\begin{equation}
c^{(2)}(\mathbf{r}_{1},\mathbf{r}_{2})=\dfrac{\delta c^{(1)}(\mathbf{r}_{1})}{\delta\rho^{(1)}(\mathbf{r}_{2})}=-\dfrac{\delta^{2}(\beta F_{\mathrm{exc}}[\rho^{(1)}])}{\delta\rho^{(1)}(\mathbf{r}_{1})\delta\rho^{(1)}(\mathbf{r}_{2})}=c^{(2)}(\mathbf{r}_{2},\mathbf{r}_{1})
\end{equation}

\begin{equation}
c^{(n)}(\mathbf{r}_{1},\ldots,\mathbf{r}_{n})=\dfrac{\delta c^{(n-1)}(\mathbf{r}_{1},\ldots,\mathbf{r}_{n-1})}{\delta\rho^{(1)}(\mathbf{r}_{n})}\label{eq:1.def-dcf}
\end{equation}

According to the definition of $F_{\mathrm{int}}$, as well as the
expression of $\delta F_{\mathrm{int}}$ in eq. (\ref{eq:delta-f-int}):
\begin{eqnarray}
\beta V_{\mathrm{int}}(\mathbf{r}) & = & \beta\dfrac{\delta F_{\mathrm{int}}[\rho^{(1)}]}{\delta\rho^{(1)}(\mathbf{r})}=\beta\dfrac{\delta F_{\mathrm{id}}[\rho^{(1)}]}{\delta\rho^{(1)}(\mathbf{r})}+\beta\dfrac{\delta F_{\mathrm{exc}}[\rho^{(1)}]}{\delta\rho^{(1)}(\mathbf{r})}\nonumber \\
 & = & \ln\left(\Lambda^{3}\rho^{(1)}(\mathbf{r})\right)-c^{(1)}(\mathbf{r})\label{eq:c1}
\end{eqnarray}

The functional derivative chain rule leads to
\begin{eqnarray}
\int\mathrm{d}\mathbf{r}_{3}\dfrac{\delta V_{\mathrm{int}}(\mathbf{r}_{1})}{\delta\rho^{(1)}(\mathbf{r}_{3})}\cdot\dfrac{\delta\rho^{(1)}(\mathbf{r}_{3})}{\delta V_{\mathrm{int}}(\mathbf{r}_{2})} & = & \int\mathrm{d}\mathbf{r}_{3}\dfrac{\delta V_{\mathrm{int}}[\rho^{(1)}(\mathbf{r}_{1})]}{\delta\rho^{(1)}(\mathbf{r}_{3})}\cdot\beta H^{(2)}(\mathbf{r}_{3},\mathbf{r}_{2})\nonumber \\
 & = & \int\mathrm{d}\mathbf{r}_{3}\left[\dfrac{1}{\rho^{(1)}(\mathbf{r}_{1})}\delta(\mathbf{r}_{1}-\mathbf{r}_{3})-c^{(2)}(\mathbf{r}_{1},\mathbf{r}_{3})\right]\cdot H^{(2)}(\mathbf{r}_{3},\mathbf{r}_{2})\nonumber \\
 & = & \delta(\mathbf{r}_{1}-\mathbf{r}_{2})
\end{eqnarray}
in addition to the definition of $H$ in eq. (\ref{eq:H-definition})
gives

\begin{equation}
h^{(2)}(\mathbf{r}_{1},\mathbf{r}_{2})=c^{(2)}(\mathbf{r}_{1},\mathbf{r}_{2})+\int\mathrm{d}\mathbf{r}_{3}\left(c^{(2)}(\mathbf{r}_{1},\mathbf{r}_{3})\rho^{(1)}(\mathbf{r}_{3})h^{(2)}(\mathbf{r}_{3},\mathbf{r}_{2})\right)\label{eq:1.oz-origine}
\end{equation}
which is called the Ornstein-Zernike (\acs{OZ}) equation. 

\section{Classical density functional theory\label{sec:Classical-density-functional}}

The density functional theory is based on two theorems :
\begin{enumerate}
\item For a given choice of $V_{N}$, $T$ and $\mu$, the intrinsic free
energy $F_{\mathrm{int}}$ is a unique functional of the equilibrium
one-particle density $\bar{n}(\mathbf{r})$, expressed by $F_{\mathrm{int}}[\bar{n}]$.
\item Let $n(\mathbf{r})$ be some arbitrary one-particle microscopic density,
and define the grand potential functional $\varOmega[n]$ as:
\begin{equation}
\varOmega[n]=F_{\mathrm{int}}[n]-\int\mathrm{d}\mathbf{r}n(\mathbf{r})V_{\mathrm{int}}(\mathbf{r})\label{eq:1.omega-p}
\end{equation}
Then the variational principle states that
\begin{equation}
\varOmega[n]\geq\varOmega[\bar{n}]
\end{equation}
with the equal sign takes at $n(\mathbf{r})=\bar{n}(\mathbf{r})$.
The differentiation of eq. (\ref{eq:1.omega-p}) with respect to $n(\mathbf{r})$
gives
\begin{equation}
\left.\frac{\delta\varOmega[n]}{\delta n(\mathbf{r})}\right|_{n=\bar{n}}=\left.\frac{\delta F_{\mathrm{int}}[n]}{\delta n(\mathbf{r})}\right|_{n=\bar{n}}-V_{\mathrm{int}}(\mathbf{r})=0\label{eq:1.26}
\end{equation}
The fact that the right hand vanishes at equilibrium is agreed with by
eq. (\ref{eq:delta-f-int}).
\end{enumerate}
The solvation free energy functional $\mathcal{F}$\marginpar{Here the character $\mathcal{F}$ is used for ``free-energy functional'';
it is a free energy of grand ensemble, but differs from Helmholtz free
energy $F$. However, it can be proven that all the free energies
become the same when the fluctuations in $N$ and $V$ are negligible
\citep{ensemble_thermo}.} is defined as the difference between the grand potential functional
of the solution system $\varOmega[n]$ and of the correspondent reference
bulk solvent at equilibrium $\varOmega[\bar{n}_{0}]$:
\begin{equation}
\mathcal{F}[n]=\varOmega[n]-\varOmega[\bar{n}_{0}]\label{eq:1.def.functional}
\end{equation}

As the external potential is absent for bulk solvent, we define:
\begin{eqnarray}
\mathcal{F}_{\mathrm{int}}[n] & = & \mathcal{F}[n]-\int\mathrm{d}\mathbf{r}n(\mathbf{r})V_{\mathrm{ext}}(\mathbf{r})\\
 & = & F_{\mathrm{int}}[n]-F_{\mathrm{int}}[\bar{n}_{0}]-\mu\int\mathrm{d}\mathbf{r}\Delta n(\mathbf{r})\nonumber \\
 & = & k_{\mathrm{B}}T\int\mathrm{d}\mathbf{r}n(\mathbf{r})\left[\ln\left(\Lambda^{3}n(\mathbf{r})\right)-1\right]+F_{\mathrm{exc}}\left[n(\mathbf{r})\right]\\
 &  & -k_{\mathrm{B}}T\int\mathrm{d}\mathbf{r}\bar{n}_{0}\left[\ln\left(\Lambda^{3}\bar{n}_{0}\right)-1\right]-F_{\mathrm{exc}}\left[\bar{n}_{0}\right]-\mu\int\mathrm{d}\mathbf{r}\Delta n(\mathbf{r})\nonumber 
\end{eqnarray}
where $\Delta n(\mathbf{r})=n(\mathbf{r})-\bar{n}_{0}$.

If we write the external free energy $F_{\mathrm{exc}}\left[n(\mathbf{r})\right]$
in Taylor expansion around $\bar{n}_{0}$:
\begin{eqnarray}
F_{\mathrm{exc}}\left[n\right] & \equiv & F_{\mathrm{exc}}\left[\bar{n}_{0}\right]+\int\mathrm{d}\mathbf{r}\left.\frac{\delta F_{\mathrm{exc}}\left[n\right]}{\delta n(\mathbf{r})}\right|_{n=\bar{n}_{0}}\Delta n(\mathbf{r})\nonumber \\
 &  & +\frac{1}{2}\int\mathrm{d}\mathbf{\mathbf{r}}_{1}\mathrm{d}\mathbf{r}_{2}\left.\frac{\delta^{2}F_{\mathrm{exc}}\left[n\right]}{\delta n(\mathbf{r}_{1})\delta n(\mathbf{r}_{2})}\right|_{n=\bar{n}_{0}}\Delta n(\mathbf{r}_{1})\Delta n(\mathbf{r}_{2})+\mathcal{O}(\Delta n^{3})\nonumber \\
 & = & F_{\mathrm{exc}}\left[\bar{n}_{0}\right]-k_{\mathrm{B}}T\int\mathrm{d}\mathbf{r}c_{0}^{(1)}(\mathbf{r})\Delta n(\mathbf{r})\nonumber \\
 &  & -\frac{k_{\mathrm{B}}T}{2}\int\mathrm{d}\mathbf{\mathbf{r}}_{1}\mathrm{d}\mathbf{r}_{2}c_{0}^{(2)}(\mathbf{r}_{1},\mathbf{r}_{2})\Delta n(\mathbf{r}_{1})\Delta n(\mathbf{r}_{2})+\mathcal{O}(\Delta n{}^{3})\label{eq:taylor-fexc}
\end{eqnarray}
where $c_{0}^{(n)}(\mathbf{r})$ is the corresponding bulk \acs{DCF}
at equilibrium defined in eq. (\ref{eq:1.def-dcf}). According to
eq. (\ref{eq:c1}):
\begin{equation}
c_{0}^{(1)}(\mathbf{r})=\ln\left(\Lambda^{3}\bar{n}_{0}\right)-\beta\mu
\end{equation}
we can find
\begin{eqnarray}
\mathcal{F}_{\mathrm{int}}[n] & = & k_{\mathrm{B}}T\int\mathrm{d}\mathbf{r}\left[n(\mathbf{r})\ln\left(\dfrac{n(\mathbf{r})}{\bar{n}_{0}}\right)-n(\mathbf{r})+\bar{n}_{0}\right]\\
 &  & -\frac{k_{\mathrm{B}}T}{2}\int\mathrm{d}\mathbf{\mathbf{r}}_{1}\mathrm{d}\mathbf{r}_{2}c_{0}^{(2)}(\mathbf{r}_{1},\mathbf{r}_{2})\Delta n(\mathbf{r}_{1})\Delta n(\mathbf{r}_{2})+\mathcal{O}(\Delta n{}^{3})\nonumber 
\end{eqnarray}

Therefore, if we define:
\begin{equation}
\mathcal{F}_{\mathrm{id}}[n]=k_{\mathrm{B}}T\int\mathrm{d}\mathbf{r}\left[n(\mathbf{r})\ln\left(\dfrac{n(\mathbf{r})}{\bar{n}_{0}}\right)-n(\mathbf{r})+\bar{n}_{0}\right]
\end{equation}
\begin{equation}
\mathcal{F}_{\mathrm{exc}}[n]=-\frac{k_{\mathrm{B}}T}{2}\int\mathrm{d}\mathbf{\mathbf{r}}_{1}\mathrm{d}\mathbf{r}_{2}c_{0}^{(2)}(\mathbf{r}_{1},\mathbf{r}_{2})\Delta n(\mathbf{r}_{1})\Delta n(\mathbf{r}_{2})+\mathcal{O}(\Delta n{}^{3})\label{eq:fexc-complet}
\end{equation}
\begin{equation}
\mathcal{F}_{\mathrm{ext}}[n]=\int\mathrm{d}\mathbf{r}n(\mathbf{r})V_{\mathrm{ext}}(\mathbf{r})
\end{equation}
the free energy functional can be written as:
\begin{equation}
\mathcal{F}[n]=\mathcal{F}_{\mathrm{int}}+\mathcal{F}_{\mathrm{ext}}=\mathcal{F}_{\mathrm{id}}+\mathcal{F}_{\mathrm{exc}}+\mathcal{F}_{\mathrm{ext}}
\end{equation}

Up to this point a brilliant approach has been built, in that for a given
choice $V_{N}$, $T$ and $\mu$, one can obtain the equilibrium density
of solvent $\bar{n}(\mathbf{r})$ by minimizing the free energy functional:
\begin{equation}
\left.\frac{\delta\mathcal{F}[n]}{\delta n(\mathbf{r})}\right|_{n=\bar{n}}=0\label{eq:1.59}
\end{equation}

Note that the two terms $\mathcal{F}_{\mathrm{id}}[n]$ and $\mathcal{F}_{\mathrm{ext}}[n]$
are physically exact, while the excess term $\mathcal{F}_{\mathrm{exc}}[n]$,
which can be rewritten as:
\begin{equation}
\mathcal{F}_{\mathrm{exc}}[n]=-\frac{k_{\mathrm{B}}T}{2}\int\mathrm{d}\mathbf{\mathbf{r}}_{1}\mathrm{d}\mathbf{r}_{2}C(\mathbf{r}_{1},\mathbf{r}_{2})\Delta n(\mathbf{r}_{1})\Delta n(\mathbf{r}_{2})
\end{equation}
depends on the exact correlation function $C(\mathbf{r}_{1},\mathbf{r}_{2})$
which is a priori unknown.

If we ignore the three-body and higher order terms in eq. (\ref{eq:fexc-complet}),
$C(\mathbf{r}_{1},\mathbf{r}_{2})$ becomes that of the homogeneous
reference fluid, which only depends on the relative distance, i.e.
$c^{(2)}(\mathbf{r}_{1},\mathbf{r}_{2})=c(r_{12})$, so that
\begin{equation}
\mathcal{F}_{\mathrm{exc}}\left[n\right]\simeq-\frac{k_{\mathrm{B}}T}{2}\int\mathrm{d}\mathbf{\mathbf{r}}_{1}\mathrm{d}\mathbf{r}_{2}c(r_{12})\Delta n(\mathbf{r}_{1})\Delta n(\mathbf{r}_{2})
\end{equation}
This was called the homogenous reference fluid (\acs{HRF}) approximation.
The generalization to a molecular, non-spherical solvent for which
orientations matter is described in $\mathsection$\ref{chpt:iem-mdft}.

\section{Integral equation theory}

Similar to the \acs{DFT} approach which aims to find the equilibrium
solvent density $\rho^{(1)}=\bar{n}$ and the free energy functional
$\mathcal{F}$, the integral equation theory (\acs{IET}) can also
give structural and energetic informations by solving a pair of integral
equations of $h^{(2)}(\mathbf{r}_{1},\mathbf{r}_{2})$ and $c^{(2)}(\mathbf{r}_{1},\mathbf{r}_{2})$
to find the pair distribution function $g$ and the difference of
correlation functions $\gamma=h-c$ which is directly linked to the
free energy. One of the relations for $h$ and $c$ is the \acs{OZ}
equation shown as eq. (\ref{eq:1.oz-origine}). Note that in $k$-space
eq. (\ref{eq:1.oz-origine}) can take advantage of the convolution
properties to give a simple product relation:
\begin{equation}
\gamma(\mathbf{k})=h(\mathbf{k})-c(\mathbf{k})=\rho\left(\gamma(\mathbf{k})+c(\mathbf{k})\right)c(\mathbf{k})\label{eq:3.oz-k}
\end{equation}

The second relation is a closure equation, which can be deduced from
eq. (\ref{eq:1.59}), giving the minimum density
\begin{equation}
\rho^{(1)}(\mathbf{r}_{1})=\rho_{0}^{(1)}\exp\left(-\beta V_{\mathrm{ext}}(\mathbf{r}_{1})+\int\mathrm{d}\mathbf{\mathbf{r}}_{2}\Delta\rho^{(1)}(\mathbf{r}_{2})c^{(2)}(\mathbf{r}_{1},\mathbf{r}_{2})+\mathcal{O}(\Delta\rho{}^{2})\right)\label{eq:1.minimize-anal}
\end{equation}
which gives, for example, when $\mathcal{O}(\Delta\rho{}^{2})=0$,
one of the simplest closure equations, the hypernetted-chain \acs{HNC}
approximation:
\begin{equation}
g(1,2)=1+h(1,2)=\exp\left[-\beta u(1,2)+h(1,2)-c(1,2)\right]
\end{equation}
Here $u$ corresponds to $V_{\mathrm{ext}}$ in eq. (\ref{eq:1.minimize-anal})
when the particles 1 and 2 are respectively the solute and solvent.

The general form of \acs{OZ} closure is:
\begin{equation}
g(1,2)=\exp\left[-\beta u(1,2)+h(1,2)-c(1,2)+b(1,2)\right]
\end{equation}
where the $b$ is the bridge function. Other closures are also possible,
such as Percus-Yevick (PY) approximation (a linear expansion of the
second exponential term in \acs{HNC}) specifically for systems with
short-range forces, or mean-spherical approximation (MSA) in the limit
of low density.

\section{Equivalence between cDFT and IET for a dilute solution system\label{sec:Equivalence-iet-mdft}}

The generalization of the \acs{OZ} equation in eq. (\ref{eq:1.oz-origine})
to $n$ components can be written as
\begin{equation}
h_{\nu\mu}(1,2)=c_{\nu\mu}(1,2)+\rho\sum_{\lambda}x_{\lambda}\int c_{\nu\lambda}(2,3)h_{\lambda\mu}(1,3)\mathrm{d}3
\end{equation}
where $x_{\nu}=N_{\nu}/N$ is the number concentration of species
$\nu\in\left[1,n\right]$.

For a two-component homogeneous solute-solvent mixture, where the
solute (M) is infinitely diluted in the solvent (S) ($x_{S}\rightarrow1$),
the coupled \acs{OZ} relations are written as
\begin{equation}
h_{\mathrm{SS}}(1,2)=c_{\mathrm{SS}}(1,2)+\rho\int h_{\mathrm{SS}}(1,3)c_{\mathrm{SS}}(2,3)\mathrm{d}3\label{eq:2oz1}
\end{equation}
\begin{equation}
h_{\mathrm{SM}}(1,2)=c_{\mathrm{SM}}(1,2)+\rho\int h_{\mathrm{SS}}(1,3)c_{\mathrm{SM}}(2,3)\mathrm{d}3\label{eq:2oz2}
\end{equation}
\begin{equation}
h_{\mathrm{MS}}(1,2)=c_{\mathrm{MS}}(1,2)+\rho\int h_{\mathrm{MS}}(1,3)c_{\mathrm{SS}}(2,3)\mathrm{d}3\label{eq:2oz3}
\end{equation}
\begin{equation}
h_{\mathrm{MM}}(1,2)=c_{\mathrm{MM}}(1,2)+\rho\int h_{\mathrm{MS}}(1,3)c_{\mathrm{SM}}(2,3)\mathrm{d}3\label{eq:2oz4}
\end{equation}

Eq. (\ref{eq:2oz1}) is the \acs{OZ} equation for bulk solvent. Eqs.
(\ref{eq:2oz2}) and (\ref{eq:2oz3}) describe the correlations between
the solute and solvents, which are equivalent. From eq. (\ref{eq:2oz3})
we can deduce eq. (\ref{eq:fexc-complet}) for the \acs{DFT} approach,
if we impose $\mathcal{O}(\Delta\rho{}^{3})=0$, i.e. the \acs{HNC}
approximation. And in \acs{IET}, eq. (\ref{eq:2oz2}) is normally
used for two-component solution. Eq. (\ref{eq:2oz4}) is rarely used.
The difficulty to solve such equation lies in finding a proper closure
equation. As the approximations like \acs{HNC} are already quantitatively
far from sufficient to describe solute-solvent correlation, it becomes
very bad for solute-solute.
