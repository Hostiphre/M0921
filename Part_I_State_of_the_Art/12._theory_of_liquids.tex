
\chapter{Statistical Mechanics of Atomic Fluids\label{chpt:statistical-mechanics}}

Statistical mechanics serves to deduce thermodynamic quantities from
the Hamiltonian of any given system. In this section, we present some
basic formalism for a classical atom-like spherical solvent model
in grand canonical ensemble ($\mu$,$V$,$T$). Firstly, we introduce
the relations between the statistical mechanics and thermodynamic
quantities. Then, we change the view to the structure of the solvent.
The two theories we use in this thesis, here refers to \acs{IET}
and \acs{MDFT}, as well as their equivalency are proven in the following
sections. The most of this sections is based on the book of Hansen
\& McDonald \citep{HANSEN_2ed,Hensen-McDonald}, and the articles
and notes of Evans \citep{Evans_1979,Evans_1984,Evans_1992}.

\section{Hamiltonian and ensemble properties}

Once we defined a spherical solvent model, of which the movement only
depends on its position and momentum $(\mathbf{r},\mathbf{p})$, the
instantaneous state (phase point, micro-state) of an $N$-particle
solvent system is specified by $3N$ coordinates $\mathbf{r}^{N}\equiv\mathbf{r}_{1},\ldots,\mathbf{r}_{N}$
and $3N$ momenta $\mathbf{p}^{N}\equiv\mathbf{p}_{1},\ldots,\mathbf{p}_{N}$.
The internal energy of particles in a system is characterized by its
Hamiltonian:
\begin{equation}
H_{N}(\mathbf{r}^{N},\mathbf{p}^{N})=K_{N}(\mathbf{p}^{N})+V_{N}(\mathbf{r}^{N})+V_{N}^{\mathrm{ext}}(\mathbf{r}^{N})
\end{equation}
where

\begin{tabular*}{1\columnwidth}{@{\extracolsep{\fill}}l>{\raggedright}p{0.9\columnwidth}}
$K_{N}(\mathbf{p}^{N})$ & $={\displaystyle \sum_{i=1}^{N}\frac{\mathbf{p}_{i}^{2}}{2m}}$ is
the kinetic energy;\tabularnewline
$V_{N}(\mathbf{r}^{N})$ & $={\displaystyle \sum_{i<j}^{N}u(\left|\mathbf{r}_{i}-\mathbf{r}_{j}\right|)+3\,\mathrm{body}+\ldots}$
is the interatomic potential energy $\mathcal{U}(\mathbf{r}^{N})$;\tabularnewline
$V_{N}^{\mathrm{ext}}(\mathbf{r}^{N})$ & $={\displaystyle \sum_{i=1}^{N}}V_{\mathrm{ext}}(\mathbf{r}_{i})$
is the potential energy arising from the interaction of the particles
with the external field (e. g. a solute).\tabularnewline
 & \tabularnewline
\end{tabular*}

The grand potential, characteristic thermodynamic state function for
the grand canonical ensemble, which depends on the chemical potential
$\mu$, the volume $V$ and the temperature $T$, is linked with the
statistical mechanics quantities with the relation:
\begin{equation}
\Omega(\mu,V,T)=-k_{\mathrm{B}}T\ln\Xi\label{eq:2}
\end{equation}
where
\begin{eqnarray}
\Xi & = & \sum_{N=0}^{\infty}\frac{e^{\beta\mu N}}{h^{3N}N!}\int\mathrm{d}\mathbf{r}^{N}\mathrm{d}\mathbf{p}^{N}e^{-\beta H_{N}(\mathbf{r}^{N},\mathbf{p}^{N})}\\
 & = & \sum_{N=0}^{\infty}\dfrac{\text{1}}{N!}\int\mathrm{d}\mathbf{r}^{N}e^{-\beta V_{N}(\mathbf{r}^{N})}\left(\prod_{i=1}^{N}\frac{e^{\beta V_{\mathrm{int}}(\mathbf{r}_{i})}}{\Lambda^{3}}\right)
\end{eqnarray}
 is the grand partition function, with $\Lambda=\left(2\pi\beta\hbar^{2}/m\right)^{-\frac{1}{2}}$
the de Broglie thermal wavelength, and
\begin{equation}
V_{\mathrm{int}}(\mathbf{r}_{i})=\mu-V_{\mathrm{ext}}(\mathbf{r}_{i})\label{eq:5}
\end{equation}
the intrinsic chemical potential. 

We can also define the intrinsic free energy: \marginpar{$N=\int\mathrm{d}\mathbf{r}\rho(\mathbf{r})$ is the number of particles
in canonical ensemble, but the formula (\ref{eq:f-int}) and (\ref{eq:delta-f-int})
are also available for grand canonical ensemble.}
\begin{eqnarray}
\mathcal{F}_{\mathrm{int}} & = & F-\int\mathrm{d}\mathbf{r}\rho(\mathbf{r})V_{\mathrm{ext}}(\mathbf{r})\nonumber \\
 & = & \Omega+\mu N-\int\mathrm{d}\mathbf{r}\rho(\mathbf{r})V_{\mathrm{ext}}(\mathbf{r})\nonumber \\
 & = & \Omega+\int\mathrm{d}\mathbf{r}\rho(\mathbf{r})V_{\mathrm{int}}(\mathbf{r})\label{eq:f-int}
\end{eqnarray}
where 
\begin{equation}
\rho(\mathbf{r})=\left\langle \varrho(\mathbf{r})\right\rangle =\left\langle \sum_{i=1}^{N}\delta(\mathbf{r}-\mathbf{r}_{1})\right\rangle 
\end{equation}
is the density profile of instantaneous density $\varrho(\mathbf{r})$
distribution.

The differential form of $\mathcal{F}_{\mathrm{int}}$ is
\begin{equation}
\delta\mathcal{F}_{\mathrm{int}}=-S\delta T+\int\mathrm{d}\mathbf{r}\delta\rho(\mathbf{r})V_{\mathrm{int}}(\mathbf{r})\label{eq:delta-f-int}
\end{equation}
with $S$ the entropy.

The internal energy of the solvent contains two contributions, one
due to the kinetic energy of the particles, $K_{N}(\mathbf{p}^{N})$,
and the other part linked to the interaction between particles, $V_{N}(\mathbf{r}^{N})$.
When the fluid is a perfect gas, which means $V_{N}=0$, it can be
easily derived from eq. (\ref{eq:2}-\ref{eq:5}) that $\mathcal{F}_{\mathrm{int}}$
has the following expression:
\begin{equation}
\mathcal{F}_{\mathrm{id}}=\beta^{-1}\int\mathrm{d}\mathbf{r}\rho(\mathbf{r})\left(\ln\left[\Lambda^{3}\rho(\mathbf{r})\right]-1\right)
\end{equation}

When interactions are accounted for, the total expression of $\mathcal{F}_{\mathrm{int}}$
can be
\begin{equation}
\mathcal{F}_{\mathrm{int}}=\mathcal{F}_{\mathrm{id}}+\mathcal{F}_{\mathrm{exc}}\label{eq:f-int-def}
\end{equation}
and the form of $\mathcal{F}_{\mathrm{exc}}$ will be detailed in
latter sections.

\section{Functional derivatives and distribution functions}

The structure of the solvent in the grand canonical ensemble can be
characterized by its $n$-particle density 
\begin{equation}
\rho^{(n)}(\mathbf{r}^{n})=\dfrac{1}{\Xi}\sum_{N=n}^{\infty}\dfrac{1}{(N-n)!}\int\mathrm{d}\mathbf{r}^{\left(N-n\right)}e^{-\beta V_{N}(\mathbf{r}^{N})}\left(\prod_{i=1}^{N}\frac{e^{\beta V_{\mathrm{int}}(\mathbf{r}_{i})}}{\Lambda^{3}}\right)\label{eq:def-rho}
\end{equation}
which means the probability to find $n$ particles in an volume element
$\mathrm{d}\mathbf{r}^{n}$. In particular, the probability to find
one particle in an volume element is the solvent density $\rho^{(1)}(\mathbf{r})=\rho(\mathbf{r})$,
that 
\begin{equation}
\rho^{(1)}(\mathbf{r})\mathrm{d}\mathbf{r}=\left\langle N\right\rangle 
\end{equation}
where $\left\langle N\right\rangle $ is the ensemble average of the
number of particle, \textcolor{red}{that is to say the average number
of particle at equilibrium.} $\rho^{(n)}(\mathbf{r}^{n})$ becomes
$\rho^{n}$ if the system is homogeneous. It can be proven that
\begin{equation}
\dfrac{\delta\Omega}{\delta V_{\mathrm{int}}(\mathbf{r})}=-\rho^{(1)}(\mathbf{r})
\end{equation}

The corresponding $n$-particle distribution function is defined as:
\begin{equation}
g^{(n)}(\mathbf{r}^{n})=\dfrac{\rho^{(n)}(\mathbf{r}^{n})}{\prod_{i=1}^{n}\rho^{(1)}(\mathbf{r}_{i})}
\end{equation}
such that $g^{(n)}(\mathbf{r}^{n})\rightarrow1$ when all pairs of
particles becomes sufficiently large.

The two-particle pair distribution function (\acs{PDF}), $g^{(2)}(\mathbf{r}_{1},\mathbf{r}_{2})$,
is one of the most important quantity in the theory of liquids. And
its corresponding pair correlation function (\acs{PCF}) is defined
as:
\begin{equation}
h^{(2)}(\mathbf{r}_{1},\mathbf{r}_{2})=g^{(2)}(\mathbf{r}_{1},\mathbf{r}_{2})-1
\end{equation}
and it vanishes when $\left|\mathbf{r}_{1}-\mathbf{r}_{2}\right|\rightarrow\infty$.

If we define the density-density correlation function as: \marginpar{For any ensemble.}
\begin{equation}
H^{(2)}(\mathbf{r}_{1},\mathbf{r}_{2})=\rho^{(1)}(\mathbf{r}_{1})\rho^{(1)}(\mathbf{r}_{2})h^{(2)}(\mathbf{r}_{1},\mathbf{r}_{2})-\rho^{(1)}(\mathbf{r}_{1})\delta(\mathbf{r}_{1}-\mathbf{r}_{2})\label{eq:H-definition}
\end{equation}
which means the correlation \citep{Correlation_function_wiki} between
the instantaneous fluctuation of particle density from its ensemble
average, it can be proven that
\begin{equation}
\dfrac{\delta\Omega^{2}}{\delta V_{\mathrm{int}}(\mathbf{r}_{1})\delta V_{\mathrm{int}}(\mathbf{r}_{2})}=-\beta H^{(2)}(\mathbf{r}_{1},\mathbf{r}_{2})=-\dfrac{\delta\rho^{(1)}(\mathbf{r}_{1})}{\delta V_{\mathrm{int}}(\mathbf{r}_{2})}
\end{equation}

As an analogue, the direct correlation function (\acs{DCF}) is defined
as the derivative of the excess free energy functional $\mathcal{F}_{\mathrm{exc}}[\rho]$:

\begin{equation}
c^{(1)}(\mathbf{r})=-\dfrac{\delta(\beta\mathcal{F}_{\mathrm{exc}}[\rho^{(1)}])}{\delta\rho^{(1)}(\mathbf{r})}\label{eq:def-dcf}
\end{equation}
\begin{equation}
c^{(2)}(\mathbf{r}_{1},\mathbf{r}_{2})=\dfrac{\delta c^{(1)}(\mathbf{r}_{1})}{\delta\rho^{(1)}(\mathbf{r}_{2})}=-\dfrac{\delta^{2}(\beta\mathcal{F}_{\mathrm{exc}}[\rho^{(1)}])}{\delta\rho^{(1)}(\mathbf{r}_{1})\delta\rho^{(1)}(\mathbf{r}_{2})}=c^{(2)}(\mathbf{r}_{2},\mathbf{r}_{1})
\end{equation}

\begin{equation}
c^{(n)}(\mathbf{r}_{1},\ldots,\mathbf{r}_{n})=\dfrac{\delta c^{(n-1)}(\mathbf{r}_{1},\ldots,\mathbf{r}_{n-1})}{\delta\rho^{(1)}(\mathbf{r}_{n})}
\end{equation}

According to the definition of $F_{\mathrm{int}}$, as well as the
expression of $\delta F_{\mathrm{int}}$ in eq. (\ref{eq:delta-f-int}),
we have
\begin{eqnarray}
\beta V_{\mathrm{int}}(\mathbf{r}) & = & \beta\dfrac{\delta F_{\mathrm{int}}[\rho^{(1)}]}{\delta\rho^{(1)}(\mathbf{r})}=\beta\dfrac{\delta F_{\mathrm{id}}[\rho^{(1)}]}{\delta\rho^{(1)}(\mathbf{r})}+\beta\dfrac{\delta F_{\mathrm{exc}}[\rho^{(1)}]}{\delta\rho^{(1)}(\mathbf{r})}\nonumber \\
 & = & \ln\left[\Lambda^{3}\rho^{(1)}(\mathbf{r})\right]+c^{(1)}(\mathbf{r})
\end{eqnarray}

The functional derivative chain rule leads to
\begin{eqnarray}
\int\mathrm{d}\mathbf{r}_{3}\dfrac{\delta V_{\mathrm{int}}(\mathbf{r}_{1})}{\delta\rho^{(1)}(\mathbf{r}_{3})}\cdot\dfrac{\delta\rho^{(1)}(\mathbf{r}_{3})}{\delta V_{\mathrm{int}}(\mathbf{r}_{2})} & = & \int\mathrm{d}\mathbf{r}_{3}\dfrac{\delta V_{\mathrm{int}}[\rho^{(1)}(\mathbf{r}_{1})]}{\delta\rho^{(1)}(\mathbf{r}_{3})}\cdot\beta H^{(2)}(\mathbf{r}_{3},\mathbf{r}_{2})\nonumber \\
 & = & \delta(\mathbf{r}_{1}-\mathbf{r}_{2})
\end{eqnarray}
which in addition to the definition of $H$ in eq. (\ref{eq:H-definition})
gives

\begin{equation}
h^{(2)}(\mathbf{r}_{1},\mathbf{r}_{2})=c^{(2)}(\mathbf{r}_{1},\mathbf{r}_{2})+\int\mathrm{d}\mathbf{r}_{3}\left(c^{(2)}(\mathbf{r}_{1},\mathbf{r}_{3})\rho^{(1)}(\mathbf{r}_{3})h^{(2)}(\mathbf{r}_{3},\mathbf{r}_{2})\right)\label{eq:oz-origine}
\end{equation}
which is called the Ornstein-Zernike (\acs{OZ}) equation. 

\section{Classical density functional theory}

The density functional theory is based on two theorems :
\begin{enumerate}
\item For a given choices $V_{N}$, $T$ and $\mu$, the intrinsic free
energy functional $\mathcal{F}_{\mathrm{int}}$ is a unique functional
of the equilibrium one-particle density $\rho^{(1)}(\mathbf{r})$,
expressed by $\mathcal{F}_{\mathrm{int}}[\rho^{(1)}]$.
\item Let $n(\mathbf{r})$ be some arbitrary one-particle microscopic density,
and define the grand-potential $\Omega[n]$ as
\begin{equation}
\Omega[n]=\mathcal{F}_{\mathrm{int}}[n]-\int\mathrm{d}\mathbf{r}n(\mathbf{r})V_{\mathrm{int}}(\mathbf{r})\label{eq:omega-p}
\end{equation}
for a fixed external potential $V_{\mathrm{ext}}$ (or intrinsic chemical
potential $V_{\mathrm{int}}$), then the variational principle states
that
\begin{equation}
\Omega[n]\geq\Omega[\rho^{(1)}]
\end{equation}
with the equal sign takes at $n(\mathbf{r})=\rho^{(1)}(\mathbf{r})$.
The differentiation of eq. (\ref{eq:omega-p}) with respect to $n(\mathbf{r})$
gives
\begin{equation}
\left.\frac{\delta\Omega[n]}{\delta n(\mathbf{r})}\right|_{n=\rho^{(1)}}=\left.\frac{\delta\mathcal{F}_{\mathrm{int}}[n]}{\delta n(\mathbf{r})}\right|_{n=\rho^{(1)}}-V_{\mathrm{int}}(\mathbf{r})=0\label{eq:26}
\end{equation}
with the right hand vanishes as eq. (\ref{eq:delta-f-int}).
\end{enumerate}
These theorems build an approach, that for a given choices $V_{N}$,
$T$ and $\mu$ and an external potential $V_{\mathrm{ext}}$, amounts
to minimize the total Helmholtz free energy functional of the solvent
\begin{equation}
\mathcal{F}[n(\mathbf{r})]=\mathcal{F}_{\mathrm{int}}+\mathcal{F}_{\mathrm{ext}}=\mathcal{F}_{\mathrm{id}}+\mathcal{F}_{\mathrm{exc}}+\mathcal{F}_{\mathrm{ext}}
\end{equation}
gives the equilibrium density of solvent $\rho^{(1)}(\mathbf{r})$,
$\mathcal{F}_{\mathrm{ext}}=\int\mathrm{d}\mathbf{r}n(\mathbf{r})V_{\mathrm{ext}}(\mathbf{r})$
defined the external functional. Noted that eq. (\ref{eq:26}) is
then equivalent to

\begin{equation}
\left.\frac{\delta\mathcal{F}[n]}{\delta n(\mathbf{r})}\right|_{n=\rho^{(1)}}-\mu=0
\end{equation}

The external term $\mathcal{F}_{\mathrm{exc}}$ of the Helmholtz free
energy functional can be given by defining a reference homogeneous
single-particle density $\rho_{0}$ of bulk solvent at equilibrium,
and the corresponding bulk /acs{DCF} defined as in eq. (\ref{eq:def-dcf})
via Taylor expansion:
\begin{eqnarray}
\mathcal{F}_{\mathrm{exc}}\left[\rho^{(1)}\right] & \equiv & \mathcal{F}_{\mathrm{exc}}\left[\rho_{0}\right]+\int\mathrm{d}\mathbf{r}_{1}\left.\frac{\delta\mathcal{F}_{\mathrm{exc}}\left[\rho^{(1)}\right]}{\delta\rho^{(1)}(\mathbf{r}_{1})}\right|_{\rho^{(1)}=\rho_{0}}\Delta\rho(\mathbf{r}_{1})\nonumber \\
 &  & +\frac{1}{2}\int\mathrm{d}\mathbf{\mathbf{r}}_{1}\mathrm{d}\mathbf{r}_{2}\left.\frac{\delta^{2}\mathcal{F}_{\mathrm{exc}}\left[\rho^{(1)}\right]}{\delta\rho^{(1)}(\mathbf{r}_{1})\mathrm{\delta}\rho^{(1)}(\mathbf{r}_{2})}\right|_{\rho^{(1)}=\rho_{0}}\Delta\rho(\mathbf{r}_{1})\Delta\rho(\mathbf{r}_{2})+\mathcal{O}(\Delta\rho^{3})\\
 & = & \mathcal{F}_{\mathrm{exc}}\left[\rho_{0}\right]-\beta^{-1}\int\mathrm{d}\mathbf{r}_{1}c^{(1)}(\mathbf{r})\Delta\rho^{(1)}(\mathbf{r}_{1})\nonumber \\
 &  & -\frac{\beta^{-1}}{2}\int\mathrm{d}\mathbf{\mathbf{r}}_{1}\mathrm{d}\mathbf{r}_{2}c^{(2)}(\mathbf{r}_{1},\mathbf{r}_{2})\Delta\rho^{(1)}(\mathbf{r}_{1})\Delta\rho^{(1)}(\mathbf{r}_{2})+\mathcal{O}(\Delta\rho{}^{3})
\end{eqnarray}
where $\Delta\rho=\rho^{(1)}-\rho_{0}$.....\textcolor{red}{{} (how
to remove the $\rho_{0}$ and $c^{(1)}(\mathbf{r})$ term ?)}

If the reference fluid is homogeneous, the DCF only depends on the
relative distance, i.e. $c^{(2)}(\mathbf{r}_{1},\mathbf{r}_{2})=c(r_{12})$,
so that
\begin{equation}
\mathcal{F}_{\mathrm{exc}}\left[\rho^{(1)}\right]\simeq-\frac{\beta^{-1}}{2}\int\mathrm{d}\mathbf{\mathbf{r}}_{1}\mathrm{d}\mathbf{r}_{2}c(r_{12})\Delta\rho(\mathbf{r}_{1})\Delta\rho(\mathbf{r}_{2})
\end{equation}
This was called the homogenous reference fluid (\acs{HRF}) approximation.
The generalization to a molecular, non-spherical solvent for which
orientations matter is described in $\mathsection$\ref{chpt:iem-mdft}.

\section{Integral equation theory}

Different with \acs{MDFT} which aims to find the equilibrium solvent
density $\rho$ and the Helmholtz free energy $\mathcal{F}$, the
integral equation theory (\acs{IET}) aims to find the pair distribution
function $g$ and the \textcolor{red}{gradient of energy $\gamma$.}
Both theories give complete information of solvation energy and structure.

\acs{IET} is about to solve a pair of integral equations of $h^{(2)}(\mathbf{r}_{1},\mathbf{r}_{2})$
and $c^{(2)}(\mathbf{r}_{1},\mathbf{r}_{2})$. One of these equations
is the \acs{OZ} equation shown as eq. (\ref{eq:oz-origine}). The
other one is a closure equation, which can be deduced from eq. (\ref{eq:26}),
giving the minimum density 

\begin{equation}
\rho^{(1)}(\mathbf{r}_{1})=\rho_{0}\exp\left(-\beta V_{\mathrm{ext}}(\mathbf{r}_{1})+\int\mathrm{d}\mathbf{\mathbf{r}}_{2}\Delta\rho^{(1)}(\mathbf{r}_{2})c_{0}^{(2)}(\mathbf{r}_{12})\right)\label{eq:minimize-anal}
\end{equation}
\textcolor{red}{(where is $\mu$? external potential?)} which gives
for example, one of the simplest closure equation, the hypernetted-chain
\acs{HNC} approximation. In the uniform case we can deduce
\begin{equation}
g(1,2)=\exp\left[-\beta u(1,2)+h(1,2)-c(1,2)\right]
\end{equation}
\textcolor{red}{(u?) }

The general relation to close the OZ equation is
\begin{equation}
g(1,2)=\exp\left[-\beta u(1,2)+h(1,2)-c(1,2)+b(1,2)\right]
\end{equation}
where the $b$ is the bridge function. Other closure are also possible,
such as Percus-Yevick (PY) approximation or mean-spherical approximation
(MSA).

\section{Equivalence between c\acs{DFT} and \acs{IET} for a dilute solution
system}

The generalization of the \acs{OZ} equation in eq. (\ref{eq:oz-origine})
to $n$ components can be written as
\begin{equation}
h_{\mathrm{\nu\mu}}(1,2)=c_{\mathrm{\nu\mu}}(1,2)+\rho\sum_{\lambda}x_{\lambda}\int c_{\mathrm{\nu\lambda}}(2,3)h_{\mathrm{\lambda\mu}}(1,3)\mathrm{d}3
\end{equation}
where $x_{\nu}=N_{\nu}/N$ is the number concentration of species
$\nu\in\left[1,n\right]$.

For a two-component homogeneous solute-solvent mixture, where the
solute (M) is infinitely diluted in the solvent (S) ($x_{S}\rightarrow1$),
the coupled \acs{OZ} relations is written as
\begin{equation}
h_{\mathrm{SS}}(1,2)=c_{\mathrm{SS}}(1,2)+\rho\int h_{\mathrm{SS}}(1,3)c_{\mathrm{SS}}(2,3)\mathrm{d}3\label{eq:2oz1}
\end{equation}
\begin{equation}
h_{\mathrm{SM}}(1,2)=c_{\mathrm{SM}}(1,2)+\rho\int h_{\mathrm{SS}}(1,3)c_{\mathrm{SM}}(2,3)\mathrm{d}3\label{eq:2oz2}
\end{equation}
\begin{equation}
h_{\mathrm{MS}}(1,2)=c_{\mathrm{MS}}(1,2)+\rho\int h_{\mathrm{MS}}(1,3)c_{\mathrm{SS}}(2,3)\mathrm{d}3\label{eq:2oz3}
\end{equation}
\begin{equation}
h_{\mathrm{MM}}(1,2)=c_{\mathrm{MM}}(1,2)+\rho\int h_{\mathrm{MS}}(1,3)c_{\mathrm{SM}}(2,3)\mathrm{d}3\label{eq:2oz4}
\end{equation}

Eq. (\ref{eq:2oz1}) is the \acs{OZ} equation for bulk solvent, and
eq. (\ref{eq:2oz4}) has no physical interest as $r_{MM}\rightarrow\infty$.
Eq. (\ref{eq:2oz2}) and (\ref{eq:2oz3}) describe the correlations
between the solute and solvents, which are equivalent. From eq. (\ref{eq:2oz3})
we can deduce eq. (\ref{eq:minimize-anal}) which is the analytical
result of eq. (\ref{eq:26}) in the \acs{MDFT}. And in \acs{IET},
eq. (\ref{eq:2oz3}) is normally used for two-component solution.
