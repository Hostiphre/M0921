
\chapter{Statistical Mechanics of Classical Fluids \label{chpt:statistical-mechanics}}

the link between statistical mechanics and thermodynamics

From pair distribution functions and interaction potential we can
derive information about solvent equilibrium state.

equilibrium statistical mechanics of classical fluids.

one component bulk fluids. thermodynamic state point is specified
by $\rho$ and temperature T. 


\section{Hamiltonian of a micro-state}

effective Hamiltonian, i.e. the theorist constructs a model fluid. 

In classical mechanics, the instantaneous state (phase point) of an
$N$-particle solvent system is specified by $3N$ coordinates $\mathbf{r}^{N}\equiv\mathbf{r}_{1},\ldots,\mathbf{r}_{N}$
and $3N$ momenta $\mathbf{p}^{N}\equiv\mathbf{p}_{1},\ldots,\mathbf{p}_{N}$.
The Hamiltonian os the system is

\begin{equation}
H_{N}(\mathbf{r}^{N},\mathbf{p}^{N})=K_{N}(\mathbf{p}^{N})+V_{N}(\mathbf{r}^{N})+V_{N}^{\mathrm{ext}}(\mathbf{r}^{N})
\end{equation}
where

\begin{tabular}{ll}
$K_{N}(\mathbf{p}^{N})$ & $={\displaystyle \sum_{i=1}^{N}\frac{\mathbf{p}_{i}^{2}}{2m}}$ is
the kinetic energy \tabularnewline
$V_{N}(\mathbf{r}^{N})$ & $={\displaystyle \sum_{i<j}^{N}u(\left|\mathbf{r}_{i}-\mathbf{r}_{j}\right|)+3\,\mathrm{body}+\ldots}$
is the interatomic potential energy $\mathcal{U}(\mathbf{r}^{N})$,
One then has a pairwise additive description of the total inter-particle
potential which leads one to define a model or effective Hamiltonian
as\tabularnewline
$V_{N}^{\mathrm{ext}}(\mathbf{r}^{N})$ & $={\displaystyle \sum_{i=1}^{N}}V_{\mathrm{ext}}(\mathbf{r}_{i})$
is the potential energy arising from the interaction of the particles
with the external field\tabularnewline
 & \tabularnewline
\end{tabular}

The distribution of phase points of systems of \textcolor{red}{the
ensemble} is described by a phase space probability density $f^{[N]}(\mathbf{r}^{N},\mathbf{p}^{N};t)$,
such that 
\begin{equation}
\int f^{[N]}(\mathbf{r}^{N},\mathbf{p}^{N};t)\mathrm{d}\mathbf{r}^{N}\mathrm{d}\mathbf{p}^{N}=1
\end{equation}
for all $t$.

subset of particles of size $n$, reduced phase space distribution
function 
\begin{equation}
f^{[n]}(\mathbf{r}^{n},\mathbf{p}^{n};t)=\dfrac{N!}{(N-n)!}\int f^{[N]}(\mathbf{r}^{N},\mathbf{p}^{N};t)\mathrm{d}\mathbf{r}^{N-n}\mathrm{d}\mathbf{p}^{N-n}=1
\end{equation}
where $\mathbf{r}^{n}\equiv\mathbf{r}_{1},\ldots,\mathbf{r}_{n}$
and $\mathbf{r}^{N-n}\equiv\mathbf{r}_{n+1},\ldots,\mathbf{r}_{N}$.
the probability of finding a subset of $n$ particles in the reduced
phase space element.... the factor is the number of ways of choosing
a subset of size $n$.

The Liouville theorem shows that the probability density is independent
of time.

The Bogoliubov–Born–Green–Kirkwood–Yvon hierarchy express $f^{(n)}$
in terms of $f^{(n+1)}$, approximation closure


\section{Time averages and ensemble averages / Partition functions and Thermodynamics}

The classical canonical partition function for a one component fluid
is given by,
\begin{equation}
Z_{N}(\beta,V)=\dfrac{h^{-dN}}{N!}\int\mathrm{d}\mathbf{r}^{N}\mathrm{d}\mathbf{p}^{N}e^{-\beta H_{N}}
\end{equation}
$d$ is the dimensionality and $V$ is the volume of the system. We
can integrate over the momenta to obtain
\begin{equation}
Z_{N}(\beta,V)=\Lambda^{-dN}Q_{N}
\end{equation}
\begin{equation}
Q_{N}=\dfrac{1}{N!}\int\mathrm{d}\mathbf{r}^{N}e^{-\beta\mathcal{U}(\mathbf{r}_{1},\ldots,\mathbf{r}_{n})}
\end{equation}
is the configurational partition function. Note that the potential
energy $\mathcal{U}(\mathbf{r}_{1},\ldots,\mathbf{r}_{n})$ may still
include an external field contribution.

The Helmholtz free energy is simply

\[
F_{N}(\beta,V)=-\beta^{-1}\ln Z_{N}
\]
which leads to entropy 
\[
S=\left(\dfrac{\partial F_{N}}{\partial T}\right)_{V}
\]
pressure
\[
p=\left(\dfrac{\partial F_{N}}{\partial V}\right)_{T}
\]
for bulk fluid

For an ideal (non-interacting) gas, where $\Phi\rightarrow0$, in
$d=3$???
\[
\beta F_{N}=\ln(N!\Lambda^{3N}V^{-N})=N\ln(\Lambda^{3}\rho)-N
\]
Stering equation... number density $\rho=N/V$ (a uniform ideal classical
gas.)

Th grand function We consider open systems with fixed temperature
$T$ and chemical potential $\mu$. The partition function for the
grand canonical ensemble is

\begin{equation}
\Xi(\beta???,\mu,T)=\sum_{N=0}^{\infty}\frac{e^{\beta\mu N}}{h^{3N}N!}\iint\mathrm{d}\mathbf{r}^{N}\mathrm{d}\mathbf{p}^{N}e^{-\beta H_{N}(\mathbf{r}^{N},\mathbf{p}^{N})}=\sum_{N=0}^{\infty}\frac{1}{N!}\int\mathrm{d}\mathbf{r}^{N}e^{-\beta V_{N}(\mathbf{r}^{N})}\left(\prod_{i=1}^{N}\frac{e^{\beta u(\mathbf{r}_{i})}}{\Lambda^{3N}}\right)
\end{equation}
with $\Lambda=(2\pi\hbar^{2}/m)^{\frac{1}{2}}$ and $u(\mathbf{r})=\mu-V_{\mathrm{ext}}(\mathbf{r})$.

The equilibrium density of probability 
\[
f(\mathbf{r}^{N},\mathbf{p}^{N};N)=\frac{1}{h^{3N}N!}\frac{1}{\Xi}e^{-\beta\left[H_{N}(\mathbf{r}^{N},\mathbf{p}^{N})-\mu N\right]}
\]


and the grand potential

\[
\Omega=-\beta^{-1}\ln\Xi(\beta???,\mu,T)
\]


which for the case of a uniform fluid, reduces to $\Omega=-pV$ . 

\[
\Omega\left[u\right]=\beta^{-1}\left\langle \ln\left(h^{3N}N!f(\mathbf{r}^{N},\mathbf{p}^{N};N)\right)+K_{N}(\mathbf{p}^{N})+V_{N}(\mathbf{r}^{N})-\sum_{i=0}^{N}u(\mathbf{r}_{i})\right\rangle 
\]


\[
u(\mathbf{r}_{i})=\int\delta(\mathbf{r}-\mathbf{r}_{i})u(\mathbf{r})\mathrm{d}\mathbf{r}
\]


\[
\rho(\mathbf{r})=\left\langle \sum_{i=1}^{N}\delta(\mathbf{r}-\mathbf{r}_{i})\right\rangle 
\]


\[
\Omega\left[u\right]=\beta^{-1}\left\langle \ln\left(h^{3N}N!f(\mathbf{r}^{N},\mathbf{p}^{N};N)\right)+K_{N}(\mathbf{p}^{N})+V_{N}(\mathbf{r}^{N})\right\rangle -\int\rho(\mathbf{r})u(\mathbf{r})\mathrm{d}\mathbf{r}
\]
free energy of Helmholtz

\[
\Omega\left[u\right]=F+\mu\left\langle N\right\rangle =F+\mu\int\rho(\mathbf{r})
\]



\section{Distribution functions}

the direct correlation function hierarchy:

\[
c^{(1)}(\mathbf{r})=-\dfrac{\delta(\beta\mathcal{F}_{\mathrm{exc}}[\rho])}{\delta\rho(\mathbf{r})}
\]
\[
c^{(2)}(\mathbf{r}_{1},\mathbf{r}_{2})=\dfrac{\delta c^{(1)}(\mathbf{r}_{1})}{\delta\rho(\mathbf{r}_{2})}=-\dfrac{\delta^{2}(\beta\mathcal{F}_{\mathrm{exc}}[\rho])}{\delta\rho(\mathbf{r}_{1})\delta\rho(\mathbf{r}_{2})}=c^{(2)}(\mathbf{r}_{2},\mathbf{r}_{1})
\]


\[
c^{(n)}(\mathbf{r}_{1},\ldots,\mathbf{r}_{n})=\dfrac{\delta c^{(n-1)}(\mathbf{r}_{1},\ldots,\mathbf{r}_{n-1})}{\delta\rho(\mathbf{r}_{n})}
\]
for bulk solvent

$\mathcal{F}_{\mathrm{exc}}[\rho]$ is the excess (over ideal) \textcolor{red}{Helmholtz
free energy} functional arising from the interactions.
