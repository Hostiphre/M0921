
\chapter{Solvation Structure\label{chpt:solvation-structure}}

In MDFT, all the information about solvation structure can be deduced
from the solvent density $\rho(\mathbf{r},\mathbf{\Omega})$. This
section presents some examples of structure which are used in later
chapters.


\section{Radial distribution function and site-site distribution function}

Radial distribution function (\acs{RDF}) and site-site distribution
function

It should be the same with 


\section{Radial polarization functions}

It should be the same with 


\section{Rotational invariant expansion}

appendix \ref{chpt:rotational-invariant-expansion}


\section{Rebuilt of density in a certain orientation}

appendix \ref{chpt:rotational-invariant-expansion}


\subsection{Radical distribution function of numeric density}

Content.


\subsection{Radical distribution function of polarization}

\[
F(\mathbf{r},\mathbf{\Omega})=\sum_{nl\nu}F_{0\nu}^{0nl}(r)\Phi_{0\nu}^{0nl}(\mathbf{r},\mathbf{\Omega})
\]


conversely

\[
F_{0\nu}^{0nl}(r)=\int\mathrm{d}\hat{\mathbf{r}}\mathrm{d}\mathbf{\Omega}F(\mathbf{r},\mathbf{\Omega})\Phi_{0\nu}^{0nl*}(\mathbf{r},\mathbf{\Omega})/\int\mathrm{d}\hat{\mathbf{r}}\mathrm{d}\mathbf{\Omega}\left\Vert \Phi_{0\nu}^{0nl}(\mathbf{r},\mathbf{\Omega})\right\Vert ^{2}
\]
with

\[
\Phi_{0\nu}^{0nl}(\mathbf{r},\mathbf{\Omega})=f_{n}\sum_{\nu'}\left(\begin{array}{ccc}
0 & n & l\\
0 & \nu' & -\nu'
\end{array}\right)R_{\nu'\nu}^{n}(\mathbf{\Omega})R_{-\nu',0}^{l}(\mathbf{\hat{r}})
\]


in particular

\begin{eqnarray*}
\Phi_{00}^{011}(\mathbf{r},\mathbf{\Omega}) & = & \sqrt{3}\left(\begin{array}{ccc}
0 & 1 & 1\\
0 & 0 & 0
\end{array}\right)R_{00}^{1}(\mathbf{\Omega})R_{00}^{1}(\mathbf{\hat{r}})\\
 & + & \sqrt{3}\left(\begin{array}{ccc}
0 & 1 & 1\\
0 & 1 & -1
\end{array}\right)R_{10}^{1}(\mathbf{\Omega})R_{-10}^{1}(\mathbf{\hat{r}})\\
 & + & \sqrt{3}\left(\begin{array}{ccc}
0 & 1 & 1\\
0 & -1 & 1
\end{array}\right)R_{-10}^{1}(\mathbf{\Omega})R_{10}^{1}(\mathbf{\hat{r}})
\end{eqnarray*}
or
\begin{eqnarray*}
\Phi_{00}^{011}(\mathbf{r},\mathbf{\Omega}) & = & -3R_{00}^{1}(\mathbf{\Omega})R_{00}^{1}(\mathbf{\hat{r}})\\
 & + & 3R_{10}^{1}(\mathbf{\Omega})R_{-10}^{1}(\mathbf{\hat{r}})\\
 & + & 3R_{-10}^{1}(\mathbf{\Omega})R_{10}^{1}(\mathbf{\hat{r}})
\end{eqnarray*}
noting 

\[
R_{00}^{1}(\mathbf{\Omega})=\cos\theta
\]


\[
R_{10}^{1}(\mathbf{\Omega})=-\frac{1}{\sqrt{2}}\sin\theta e^{-i\phi}
\]


one finds
\begin{eqnarray*}
\Phi_{00}^{011}(\mathbf{r},\mathbf{\Omega}) & = & -3R_{00}^{1}(\mathbf{\Omega})R_{00}^{1}(\mathbf{\hat{r}})\\
 & + & 3R_{10}^{1}(\mathbf{\Omega})R_{-10}^{1}(\mathbf{\hat{r}})\\
 & + & 3R_{-10}^{1}(\mathbf{\Omega})R_{10}^{1}(\mathbf{\hat{r}})\\
 & = & -3\mathbf{\Omega}\cdot\hat{\mathbf{r}}
\end{eqnarray*}


such that

\begin{eqnarray*}
F_{00}^{011}(r) & = & -\frac{1}{3}\int\\
 & = & P(\mathbf{r})\cdot\hat{\mathbf{r}}\\
\end{eqnarray*}


proportionality coefficient to be determined precisely... One thus
gets the radial projection of the polarization.

Content.
