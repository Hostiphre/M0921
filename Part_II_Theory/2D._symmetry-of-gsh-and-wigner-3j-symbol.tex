
\chapter{Properties of Wigner 3j-Symbol and GSH \label{chpt:symmetry}}

The properties of Wigner 3j-symbol and Wigner generalized spherical
harmonics (\acs{GSH}, Winger D-symbol) play a huge role in the reduction
of molecular Ornstein-Zernike equation as well as finding the relation
between rotational invariant projections. Their main properties, presented
in Messiah \citep{Messiah}, Gray \& Gubbins \citep{Gray-Gubbins}
and Edmonds \citep{Edmonds}, are listed here.

\section{Properties of Wigner 3j-Symbol}

Wigner 3j-symbols are equivalent to Clebsch-Gordon (CG) coefficients
multiplied by the phase factor:
\begin{equation}
\left(\begin{array}{ccc}
m & n & l\\
\mu & \nu & -\lambda
\end{array}\right)=\frac{\left(-\right)^{m-n+\lambda}}{\sqrt{2l+1}}<mn\mu\nu\mid l\lambda>
\end{equation}
and can be calculated with the Racah formula \citep{Messiah}.

\subsubsection*{Reality}

The 3j-symbols are real.
\begin{equation}
\left(\begin{array}{ccc}
m & n & l\\
\mu & \nu & \lambda
\end{array}\right)=\left(\begin{array}{ccc}
m & n & l\\
\mu & \nu & \lambda
\end{array}\right)^{*}
\end{equation}


\subsubsection*{Selection rules}

\begin{equation}
\left(\begin{array}{ccc}
m & n & l\\
\mu & \nu & \lambda
\end{array}\right)=0\;\mathrm{if}\,\left\{ \begin{array}{l}
\mu+\nu+\lambda=0\\
\underset{\mathrm{(triangular\,inequalities)}}{\left|m-n\right|<l<m+n}
\end{array}\right.\,\mathrm{are\,not\,meet.}\,
\end{equation}


\subsubsection*{Permutation }
\begin{enumerate}
\item Even permutation
\begin{equation}
\left(\begin{array}{ccc}
m & n & l\\
\mu & \nu & \lambda
\end{array}\right)=\left(\begin{array}{ccc}
n & l & m\\
\nu & \lambda & \mu
\end{array}\right)=\left(\begin{array}{ccc}
l & m & n\\
\lambda & \mu & \nu
\end{array}\right)
\end{equation}
\item Odd permutation
\begin{eqnarray}
\left(-\right)^{m+n+l}\left(\begin{array}{ccc}
m & n & l\\
\mu & \nu & \lambda
\end{array}\right) & = & \left(\begin{array}{ccc}
n & m & l\\
\nu & \mu & \lambda
\end{array}\right)\nonumber \\
=\left(\begin{array}{ccc}
m & l & n\\
\mu & \lambda & \nu
\end{array}\right) & = & \left(\begin{array}{ccc}
l & n & m\\
\lambda & \nu & \mu
\end{array}\right)
\end{eqnarray}
\item Simultaneous change of signs of $\mu$, $\nu$ and $\lambda$
\begin{equation}
\left(\begin{array}{ccc}
m & n & l\\
\mu & \nu & \lambda
\end{array}\right)=\left(-\right)^{m+n+l}\left(\begin{array}{ccc}
m & n & l\\
-\mu & -\nu & -\lambda
\end{array}\right)
\end{equation}
\end{enumerate}

\subsubsection*{Orthogonality}

\begin{equation}
\sum_{l=\left|m-n\right|}^{m+n}\sum_{\lambda=-l}^{l}\left(2l+1\right)\left(\begin{array}{ccc}
m & n & l\\
\mu & \nu & \lambda
\end{array}\right)\left(\begin{array}{ccc}
m & n & l\\
\mu' & \nu' & \lambda
\end{array}\right)=\delta_{\mu\mu'}\delta_{\nu\nu'}\label{eq:3j-orthogonality}
\end{equation}
\begin{equation}
\sum_{\mu=-m}^{m}\sum_{\nu=-n}^{n}\left(\begin{array}{ccc}
m & n & l\\
\mu & \nu & \lambda
\end{array}\right)\left(\begin{array}{ccc}
m & n & l'\\
\mu & \nu & \lambda'
\end{array}\right)=\left(2l+1\right)^{-1}\delta_{ll'}\delta_{\lambda\lambda'}
\end{equation}


\section{Properties of GSH}

There are many different definitions of GSH given in lectures. Here
we adopt the definition in Messiah:
\begin{equation}
R_{\mu'\mu}^{m}(\phi\theta\psi)=e^{-i\mu'\phi}r_{\mu'\mu}^{m}(\theta)e^{-i\mu\psi}\label{eq:GSH-def}
\end{equation}
where $r_{\mu\mu'}^{m}$ is the generalized Legendre polynomial (GLP),
which is real, and can be evaluated using the Wigner formula:
\begin{eqnarray}
r_{\mu'\mu}^{m}(\theta) & = & \left[\left(m+\mu'\right)!\left(m-\mu'\right)!\left(m+\mu\right)!\left(m-\mu\right)!\right]^{\frac{1}{2}}\times\nonumber \\
 &  & \sum_{i}\frac{\left(-\right)^{i}\left(\cos\theta/2\right)^{2m+\mu'-\mu-2i}\left(\sin\theta/2\right)^{2i-\mu'+\mu}}{\left(m+\mu'-i\right)!\left(m-\mu-i\right)!i!\left(i-\mu'+\mu\right)!}
\end{eqnarray}


\subsubsection*{Symmetries of $r_{\mu'\mu}^{m}(\theta)$}

\begin{equation}
r_{\mu\mu'}^{m}(\theta)=(-)^{\mu'-\mu}r_{\mu'\mu}^{m}(\theta)
\end{equation}
\begin{equation}
r_{\underline{\mu'}\underline{\mu}}^{m}(\theta)=\left(-\right)^{\mu'-\mu}r_{\mu'\mu}^{m}(\theta)
\end{equation}
\begin{equation}
r_{\mu'\mu}^{m}(\theta)=r_{\mu\mu'}^{m}(-\theta)
\end{equation}
\begin{equation}
r_{\mu'\mu}^{m}(\theta+\pi)=(-)^{m+\mu}r_{\mu'\underline{\mu}}^{m}(\theta)
\end{equation}
where $\underline{\mu}\equiv-\mu$.

\subsubsection*{Symmetries of $R_{\mu'\mu}^{m}(\phi\theta\psi)$}

\begin{equation}
R_{\mu'\mu}^{m}(\phi\theta\psi)=\left(-\right)^{\mu'-\mu}R_{\underline{\mu'}\underline{\mu}}^{m*}(\phi\theta\psi)\label{eq:symm-gsh-1}
\end{equation}
\begin{equation}
R_{\mu'\mu}^{m}(\phi\theta\psi)=\left(-\right)^{\mu'-\mu}R_{\mu\mu'}^{m*}(\phi\theta\psi)
\end{equation}
\begin{equation}
R_{\mu'\mu}^{m}(\phi\theta\psi)=(-)^{m+\mu'}R_{\underline{\mu'}\mu}^{m}(-\phi,\theta+\pi,\psi)=(-)^{m+\mu}R_{\mu'\underline{\mu}}^{m}(\phi,\theta+\pi,-\psi)
\end{equation}


\subsubsection*{Unitarity and orthogonality}

\begin{equation}
\sum_{\mu'}R_{\mu'\mu}^{m}(\phi\theta\psi)R_{\mu'\mu''}^{m*}(\phi\theta\psi)=\delta_{\mu\mu''}
\end{equation}
\begin{equation}
\sum_{\mu}R_{\mu'\mu}^{m}(\phi\theta\psi)R_{\mu''\mu}^{m*}(\phi\theta\psi)=\delta_{\mu'\mu''}
\end{equation}
\begin{equation}
\sum_{m\mu'\mu}R_{\mu'\mu}^{m}(\phi\theta\psi)R_{\mu'\mu'}^{m*}(\phi'\theta'\psi')=\delta_{\phi\phi'}\delta_{\theta\theta'}\delta_{\psi\psi'}
\end{equation}
\begin{equation}
\frac{1}{8\pi^{2}}\int\mathrm{d}\cos\theta\mathrm{d}\phi\mathrm{d}\psi R_{\mu'\mu}^{m}(\phi\theta\psi)R_{\nu'\nu}^{n*}(\phi\theta\psi)=\frac{\delta_{mn}\delta_{\mu'\nu'}\delta_{\mu\nu}}{2n+1}\label{eq:gsh-orthogonality}
\end{equation}


\subsubsection*{$r_{\mu'\mu}^{m}(\theta)$ in terms of $\cos\theta$ and $\sin\theta$}
\begin{enumerate}
\item If $\left(-\right)^{\mu'+\mu}=+1$, $r_{\mu'\mu}^{m}(\theta)$ is
a polynomial of degree $m$ in $\cos\theta$.
\item If $\left(-\right)^{\mu'+\mu}=-1$, $r_{\mu'\mu}^{m}(\theta)/\sin\theta$
is a polynomial of degree $(m-1)$ in $\cos\theta$.
\end{enumerate}

\subsubsection*{Rotation and product}

\begin{equation}
R_{\mu'\mu}^{m}(\boldsymbol{\omega})=\sum_{\chi}R_{\mu'\chi}^{m}(\boldsymbol{\omega_{2}})R_{\chi\mu}^{m}(\boldsymbol{\omega_{1}})
\end{equation}
where $\boldsymbol{\omega}$ is the result of the successive application
of $\boldsymbol{\omega}_{1}$ and $\boldsymbol{\omega_{2}}$ in order.

\begin{equation}
\begin{array}{c}
{\displaystyle R_{\chi\mu}^{m}(\boldsymbol{\omega})=\sum_{\mu'}R_{\mu'\chi}^{m*}(\hat{\mathbf{k}})R_{\mu'\mu}^{m}(\mathbf{\Omega})}\\
{\displaystyle R_{\mu'\mu}^{m}(\mathbf{\Omega})=\sum_{\chi}R_{\chi\mu'}^{m*}(\hat{\mathbf{k}}^{-1})R_{\chi\mu}^{m}(\boldsymbol{\omega})=\sum_{\chi}R_{\mu'\chi}^{m}(\hat{\mathbf{k}})R_{\chi\mu}^{m}(\boldsymbol{\omega})}
\end{array}\label{eq:gsh-rotation}
\end{equation}


\subsubsection*{Composition relation for \acs{GSH}s}

\subsubsection*{
\begin{equation}
\sum_{\mu'\nu'\lambda'}\left(\protect\begin{array}{ccc}
m & n & l\protect\\
\mu' & \nu' & \lambda'
\protect\end{array}\right)R_{\mu'\mu}^{m}(\phi\theta\psi)R_{\nu'\nu}^{n}(\phi\theta\psi)R_{\lambda'\lambda}^{l}(\phi\theta\psi)=\left(\protect\begin{array}{ccc}
m & n & l\protect\\
\mu & \nu & \lambda
\protect\end{array}\right)\label{eq:gg.a91}
\end{equation}
}

\section{Convention of GSH\label{sec:Convention-of-GSH}}

The convention of \acs{GSH} in books and articles used in this thesis
are different. In Messiah \citep{Messiah} and Gray \& Gubbins \citep{Gray-Gubbins},
it is defined as in eq. (\ref{eq:GSH-def}). In Edmonds \citep{Edmonds},
it is defined as:
\begin{equation}
D_{\mu'\mu}^{m}(\phi\theta\psi)=e^{i\mu'\psi}d_{\mu'\mu}^{m}(\theta)e^{i\mu\phi}\label{eq:GSH-def-1}
\end{equation}
which can be seen as the inverse rotation matrix of $R_{\mu'\mu}^{m}$.

In Blum \citep{Blum_I,Blum_II}, the equation 
\begin{equation}
D_{m0}^{l}(\phi\theta\psi)=\left(-\right)^{m}\left(\dfrac{4\pi}{2l+1}\right)^{\frac{1}{2}}Y_{m}^{l}(\theta\phi)
\end{equation}
is adopted, that means it shares the same definition as Edmonds, where
\begin{equation}
R_{\mu'\mu}^{m}(\phi\theta\psi)=D_{\mu\mu'}^{m*}(\phi\theta\psi)\label{eq:GSH-def-2}
\end{equation}

In Fries \& Patey \citep{Fries_Patey_1985}, the definition of Messiah
is used.
