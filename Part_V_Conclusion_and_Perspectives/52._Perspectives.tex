
\chapter{Perspectives\label{chpt:perspectives}}

Due to the difficulty of this thesis and my mental state, there are %I recommend replacing 'my mental state' with 'other factors'. 
a great deal of unfinished work and theory. Here I list a few such areas to give a better idea.

\section{Bug report}

bug in gamma test

rotational invariant projection

\section{Reduce memory use in MDFT}

Computing performance of parallel code

Node-level Parallelization

OpenMP, scalability with respect to the number of threads

Parallelization on Several Nodes

MPI, scalability with respect to the number of nodes

As shown in chapter III, memory leaks can cause divergence and other
problems.

\subsection{Pass to simple precision}

L-BFGS-B is in double precision

\subsection{MPI of the L-BFGS-B minimizer}

As the code is a blackbox, in Fortran 77, it is difficult to parallelize
it. 

OpenMP, MPI provide the possibility to go beyond the memory limit.
Due to the complexity of minimizer L-BFGS, this process is only added
on the part of $\mathcal{F}_{\mathrm{exc}}$ evaluation. Tests of
performance stability with respect to both threads and nodes are made.

\section{Polarizable solute}

Vext variable

\section{MDFT Viewer}

This thesis contains a segment on visualization.

Viewer is an important part of code development, which provides beautiful
visualization and easy analysis, which helps to popularize the code. GaussViewer
is a good example.

The popular language of visualization is c++, OpenDM, ...

\section{Classical SCF method}

To be more compatible with Gaussian, {[}Jensen{]}

This is only an idea, the mathematical deduction is not fully verified.

\section{Other branches of development about MDFT}

3-body, polarization, 
