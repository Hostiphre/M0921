
\chapter{Perspectives\label{chpt:perspectives}}

Due to the difficulty of this thesis and my mental state, there are
a lot of un finished work and theory. Here lists some to give un idea.

\section{Bug report}

bug in gamma test

rotational invariant projection

\section{Reduce memory use in MDFT}

Computing performance of parallel code

Node-level Parallelization

OpenMP, scalability with respect to the number of thread

Parallelization on Several Nodes

MPI, scalability with respect to the number of node

As shown in chapter III, memory leak can cause divergence and other
problem.

\subsection{Pass to simple precision}

L-BFGS-B is in double precision

\subsection{MPI of the L-BFGS-B minimizer}

As the code is a blackbox, in Fortran 77, it is difficult to parallelize
it. 

OpenMP, MPI giving the possibility to go beyond the memory limit.
Due to the complexity of minimizer L-BFGS, this process is only added
on the part of $\mathcal{F}_{\mathrm{exc}}$ evaluation. Tests of
performance stability with respect to both threads and nodes are made.

\section{Polarizable solute}

Vext variable

\section{MDFT Viewer}

This thesis contains a part of visualization

Viewer is an important part of code developing, which provide beautiful
visualization and easy analyzing helps to popularize the code. GaussViewer
is a good example.

The popular language of visualization is c++, OpenDM, ...

\section{Classical SCF method}

To be more compatible with Gaussian, {[}Jensen{]}

This is only an idea, the mathematical deduction is not fully verified.

\section{Other branches of development about MDFT}

3-body, polarization, 
