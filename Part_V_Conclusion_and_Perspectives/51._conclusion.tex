
\chapter{Conclusion\label{chpt:conclusion}}

In this thesis, we have firstly improved the original angular integration
algorithm developed previously in the group, which uses \acs{FFT}
to deal with the spatial convolution, and a direct integration over
angles. For practical reasons, this initial approach was limited to
linear or pseudo-linear solvents in order to reduce the number of
angular variables. In this work, it has been extended to deal with
arbitrary 3D molecular solvents, which is mainly about writing out
the expression of the more complicated rotation matrix which corresponds
to the 3-Euler-angle case. 

A new algorithm by \acf{GSH} expansion, which aims to produce the
same result as the algorithm mentioned above for molecular solvent
but much faster, is then built for the treatment of the angular convolution
part, inspired by Blum's $\chi$-reduction of the \acs{MOZ} equation.
This algorithm takes advantage of the rotational invariance, separating
the \acs{OZ} equation in irreducible terms according to a series
of $\chi$ values; those can be seen as an analogue of the $\mathbf{k}$-vectors
in \acs{FFT} transform that takes advantage of translational invariance.

Theoretically and practically the new algorithm has been proven much
faster than the previous one, with an acceptable accuracy lost.

The tests that have been performed show that the proposed methods
are suitable for fast and accurate calculations of solvation free
energies. The accuracy was assessed by comparing the results to a
mathematical equivalent 1D-\acf{IET} code for simple (spherical or
linear) solutes. For the solvent structure, the relatively loose cubic
spatial grid that has to be adopted in \acf{MDFT} (3-4 points per
Angstrom) prevents the theory to produce radial curves that are as
smooth as those produced by \acs{IET} for simple solutes. Furthermore
\acf{MDFT} has much more capacity to deal with complex 3D solutes
and produce the 3D solvent structure on a regular grid.

Compared to \acs{MD} and/or experimental results, it is shown that
the new method works better in most cases than the previously implemented
``dipolar-like'' method. Concerning the solvent structure, the agreement
with \acs{MD} is far from perfect but can be qualified as satisfactory
for solute molecules with limited hydrogen-bonding to the solvent.
For small ions (in particular anions), or H-bonded molecules such
as water itself or alcohols, the \acs{HRF}/\acs{HNC} theory has
to be improved or corrected. Concerning thermodynamic properties such
as solvation free energies, some small monovalent ions have been
tested in this purpose. In this case we showed that it is very important
to account for finite size effects through two types of corrections:
a Madelung correction reminiscent of the Born correction for spherical
systems, scaling as $q^{2}/L$, and a much less intuitive correction
scaling as the ion charge $q$ and accounting for the proper treatment
of boundary charges in periodic systems. The tests need to be
extended to relevant databanks of organic molecules in order to fully
assess the relevance of the method. One can anticipate that it will
require the introduction of ad-hoc pressure corrections, or three-body
correction terms in the functional, in order to compensate for the overly
high pressure given by the \acs{HRF}/\acs{HNC} functional, a
defect that has been pointed out previously.

In short, the new algorithm is now ready for chemical applications
usage. However there are still some minor problems needing to be addressed.
Firstly, there are some slight incompatibilities within
the theory and through the implementation, notably the $\gamma$ problem
and the conjugate form of \acs{DCF} which is explained as clearly
as possible in the main text. There are also some sign and normalization
factor issues of the projections $f_{\mu\nu}^{mnl}$. The grid dependence
of free energy is somewhat worrying; it might be due to a slight
deficiency in the calculation of the external potential present in
the initial version of the \acf{MDFT} code used. Such an issue is not
worrying as this thesis mainly treats the excess $\mathcal{F}_{\mathrm{exc}}$
term of the functional.
