
\chapter{Introduction\label{chpt:introduction}}

This thesis details the development of an original numerical toolkit for
physical chemists and structural biologists based on the molecular
density functional theory (\acs{MDFT}), which makes it possible to
predict the solvation properties of arbitrary molecular objects in arbitrary molecular solvents
(mainly water) efficiently and with microscopic accuracy. This introduction will aim to illuminate the objective
of this thesis, it explains why people are interested in the nature
of solvation, and where we are in terms of the computing trends in solvation
simulations.


\section{Simulation of solvent effects}

Solvation is a fundamental phenomenon in chemistry. The chemical behavior
of numerous systems strongly depends on the nature of solvency, including popular issues like metal-organic reacting centers
\citep{Mn-oxo,PCET}, or pharmaceutical etudes \citep{drug_1_Perlovich,drug_2_Perlovich,drug_3}.
The solvation properties required by etudes %Etude? I only know this in terms of music. Does it have scientific applications?
are highly variable, such
as the Gibbs free energy of solvation, solubility, partition coefficient,
saturated vapor pressure, pH value, the 3D solvation structure,
etc. Overall, the interest in these solvation properties reaches into
many domains such as chemistry, biochemistry, pharmaceuticals, medicine, and
environmental and agrochemical industries. Unlike the well-studied
quantum mechanics (\acs{QM}) for chemical interaction and macroscopic
finite element model for physical processes, the theories of solvation
are quite variable and still under development, owing to the ambiguous
compromise between accuracy and computing cost. In a word,
the studies in this domain are quite vibrant.

\begin{figure}[h]
\centering{}\textcolor{red}{}%
\begin{minipage}[t]{1\textwidth}%
\begin{center}
\includegraphics[width=1\columnwidth]{_figure/solvation}\caption[The solvation process]{The solvation process.\label{fig:Process-of-solvation} A thermodynamic
system, whose properties only depend on the initial and final states,
can go through different paths. The physical process of solvation
(left path) takes the solute from vacuum into bulk solvent, progressively
passing through the vacuum-liquid interface. Theoretically, the solvation
energy is defined as the energy consumed in such a process. In theoretical
studies, the process can be decomposed to some artificial unphysical
process (right path), involving the growth of an uncharged solute-sized
cavity within the bulk solvent, the transfer of the solute charge
distribution from vacuum into cavity, and the interaction between
the solute and solvent.}

\par\end{center}%
\end{minipage}
\end{figure}


To change a phenomenon to a model, we must first understand its process.
Solvation is defined as the process of moving a molecule from the
gas phase (or vacuum) to a condensed phase (figure \ref{fig:Process-of-solvation}),
which builds a stabilizing interaction with the solute (or solute
moiety like protein residues, interfaces, etc.) \citep{iupac}. Such
interactions are mostly classical, involving electrostatic
and van der Waals forces, with additional specific chemical effects
such as hydrogen bond formation, and quantic effects for some small
solvents whose vibrational or rotational energy states are at the same
magnitude as $k_{\mathrm{B}}T$, etc. \citep{Gray-Gubbins}.

As not all kinds of interactions are important in applications, according
to the usage, different models and methods have been developed.

For most of the 20th century \citep{Cramer_1999}, the study of
solvation effects has been dominated by continuum (implicit) models,
which depend upon the dielectric constants and are not costly
in terms of computation resources. They provide an accurate way to treat the
strong, long-range electrostatic interactions which dominate many
solvation phenomena, but lack detailed information on the first
solvation shell. The latter, which mainly includes the cavity formation
energy and solute-solvent van der Waals interactions, is often rudely
treated by introducing an artificial form of cavity that links to the
form of solute. The methods for testing electrostatic interactions include
generalized Born model, or for better estimates via Poisson-Boltzmann
calculations. These are widely integrated within \acs{QM} simulations
of the solvent by adding extra solvation terms onto the Fock or Kohn-Sham
operator \citep{Jensen,scrf,Tomasi_1994_implicit_model}. However,
the improper treatment of the first shell, where the microscopic interactions
are primarily located, often introduce potentially huge errors in free
energy evaluation, especially for polar solvents (like water), despite
the accuracy that the \acs{QM} calculation alone can achieve. Therefore,
classical molecular simulations, which describe the individual solvent
molecules (explicit), particularly the molecular dynamics (\acs{MD})
and Monte Carlo method (\acs{MC}), became the alternative solution
during the last few decades. They generate trajectories and configurations,
then estimate free energy changes by statistical mechanical techniques,
such as free energy perturbation (FEP) theory or thermodynamic integration
(TI) \citep{Jorgensen_1995_MC}. These calculations are very demanding
on computing cost, due to the need for many (hundreds or thousands) solvent molecules to form a realistic model.

Recently, a third domain of theory to describe solvents based on the
statistical mechanics of fluids has been growing rapidly. It is generally
called liquid theory, involving mainly the integral equation theory
(\acs{IET}), and the classical density functional theory for liquids.
These approaches are capable of giving the molecular nature of the first-shell,
but without calculating all the instantaneous micro-states with respect
to time, which can be integrated over positions and momentum theoretically.
Therefore, they are of faster magnitudes than those simulations done by
micro-states.

The integral equation theory (\acs{IET}) focuses on solving the Ornstein-Zernike
(\acs{OZ}) equation with a specific closure equation \citep{Hensen-McDonald,Gray-Gubbins}.
It was firstly limited to so-called ``simple liquids'' - a system
of spherical particles. Separately in 1971, Chandler and Andersen \citep{Chandler_1972_RISM}
developed the reference interaction site model (\acs{RISM}), which
discretizes the distribution and correlation functions into a site-site
set of functions, and solves the \acs{OZ} equation in a matrix \citep{hirata_molecular_2004}.
Another party, Blum \citep{Blum_I,Blum_II}, Fries and Patey \citep{Fries_Patey_1985}
extendrf the \acs{OZ} equation to molecular cased, where the distribution
and correlation functions depend on both position and orientation.
In their theory, the orientation part of yjr \acs{OZ} equation is simplified
by expanding the distribution and correlation functions on Wigner
generalized spherical harmonics.

The classical density functional theory approach deals with inhomogeneous
liquids, and uses the same variation principle and minimization
strategy \citep{mermin_thermal_1965,Evans_1979,Hansen_1987} as electronic
density functional theory \acs{DFT} that treats electric interactions
and has great success in computational chemistry. It gives the Helmholtz
free energy and the equilibrium solvent density by minimizing the
free energy functional of the solvent density in the presence of a
given external potential. Borgis and collaborators \textcolor{red}{{[}too
many ref{]}} have recently generalized it into a molecular case, named
molecular density functional theory (\acs{MDFT}), where the solvent
density depends on both position and orientation, $\rho(\mathbf{r},\mathbf{\Omega})$.
The main theoretical difficulty lies in the definition of well-funded
and reliable functionals of the excess free energy $\mathcal{F}_{\mathrm{exc}}\left[\rho\right]$,
according to the geometric complexity of the solvent molecule. Some
recent research has shown that it is compatible with linear solvents
like acetonitrile, but still have found little non-satisfaction with the
most complex solvent, i. e. water. \acs{MDFT} can be proven to be
mathematically equivalent to the two-component molecular \acs{IET}.

The majority of work of all these theories has been focused on water,
since it is one of the most difficult systems to model due to its
molecular geometry, ineligible multi-body interaction, quantum effect,
and hydrogen bonds. The importance of including instantaneous polarization
in potential functions is also an issue \citep{polarisable_1,polarisable_2}.
However, since polarizable force fields are not yet in common use,
the simulations by micro-states and the liquid theory which feed on
force fields also have their own limits compared to the continuum model,
which can be polarizable. The advantages and disadvantages of each
branch of theory are listed in table \ref{tab:Theories-of-solvation}.

\begin{table}[h]
\begin{centering}
\begin{tabular}{ccccc}
\toprule 
\tableheadline{Theory} & \tableheadline{Speed} & \tableheadline{Long-Range} & \tableheadline{First-Shell} & \tableheadline{Polarizable Solvent}\tabularnewline
\midrule
Continuum model & fast & yes & no & fully\tabularnewline
Simulation by time & costly & yes & yes & partially, very costly\tabularnewline
Liquid theory & fast & yes & yes & partially\tabularnewline
\bottomrule
\end{tabular}
\par\end{centering}

\caption{Theories of solvation simulation\label{tab:Theories-of-solvation}}
\end{table}


This thesis consists of the development of the \acs{MDFT}, focusing
on the generalization and algorithmic acceleration of the excess free
energy functional $\mathcal{F}_{\mathrm{exc}}$ evaluation under homogenous
reference fluid (\acs{HRF}) approximation, which will be discussed
in detail in later chapters. 


\section{Scope of this thesis}

Chapter I reviews a selection of models and methods to the solvent
effect. It includes the popular continuum model, the basics of
liquid theory, as well as its two frontier research domains, \acs{IET}
and \acs{MDFT}. The code structure of \acs{MDFT}, which all the
development in this thesis is based on, is also presented. There is
also a brief introduction to \acs{MD} and \acs{MC}, as well as the
generation of direct correlation function (\acs{DCF}) used in this
thesis by such methods. 

Chapter II presents all the theory developed and newly used in this
thesis. Specifically, two algorithms of excess energy functional
evaluation are proposed. One is an extension of the previous algorithm,
while the other is a new algorithm that combines the molecular \acs{OZ} equation
treatment of angular parts with MDFT. The output solvation properties
are mainly free energy and solvent structure.

Chapter III takes note of all the implementation results, then divides them
into two aspects: the ``accuracy'', which involves comparisons between
algorithms with \acs{IET} and \acs{MD} results; and the ``efficiency'',
which evaluates the computing cost of the code, both in sequential
and parallelized versions.

Chapter VI gives some application to ions and molecules.
