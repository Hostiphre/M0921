
\chapter{Introduction\label{chpt:introduction}}

This thesis aims to develop an original numerical toolkit for physical
chemists and structural biologists based on the molecular density
functional theory (\acs{MDFT}), which makes it possible to predict
the solvation properties of arbitrary molecular objects in arbitrary
molecular solvents (mainly water) efficiently and with microscopic
accuracy. The introduction will seek to highlight the objective of
this thesis and help explain such topics as why theorists are interested
in the nature of solvation, what are the present computing trends
in solvation simulations, and where our work situates in this frame
of solvation theories.

\section{Modeling of solvent effects}

Solvation is a fundamental phenomenon in chemistry. The chemical behavior
of numerous systems strongly depends on the nature of solvation; for
example, this is the case for the reaction mechanisms in metal-organic
reacting centers \citep{Mn-oxo,PCET}, or pharmaceutical studies \citep{drug_1_Perlovich,drug_2_Perlovich,drug_3}.
The solvation properties demanded by scientific studies are highly
diverse; they include the free energy of solvation, solubility, concentration,
partition coefficient, saturated vapor pressure, pH value, the 3D
solvation structure, etc. Overall, interest in these solvation properties
touches many fields of study such as chemistry and biochemistry, as
well as pharmaceutical, environmental, and agrochemical industries.
Unlike the well-studied quantum mechanics (\acs{QM}) for chemical
interactions at a microscopic scale, and the finite element models
for macroscopic physical processes, the theories of solvation lie
in-between these description scales and are still under development,
owing to the ambiguous compromise between accuracy and computing cost,
and the rapid development of computer hardware which makes complicated
calculations more and more accessible. In a word, the studies in this
domain are quite vibrant.

To change a phenomenon into a model, we must first understand its
process. Solvation is defined as the process of moving a molecule
from the gas phase (or vacuum) to a condensed phase (figure \ref{fig:Process-of-solvation}),
which builds a stabilizing interaction with the solute (or solute
moiety, e.g., residues, interfaces, etc.) \citep{iupac}. Such interactions
are mostly classical interactions, involving electrostatic and van
der Waals forces; but there are also additional specific chemical
effects such as hydrogen bond formation, and quantum effects for some
small solvent molecules whose vibrational or rotational energy states
are at the same magnitude as $k_{\mathrm{B}}T$, as well as other
effects, etc.

\begin{figure}[h]
\centering{}\textcolor{red}{}%
\noindent\begin{minipage}[t]{1\textwidth}%
\begin{center}
\includegraphics[width=1\columnwidth]{_figure/solvation}\caption[The solvation process]{The solvation process.\label{fig:Process-of-solvation} A thermodynamic
system, whose properties only depend on the initial and final states,
can go through different paths. The physical process of solvation
(left path) takes the solute from vacuum into bulk solvent, progressively
passing through the vacuum-liquid interface. Theoretically, the solvation
energy is defined as the energy consumed in such a process. In theoretical
studies, the process can also be decomposed into some artificial unphysical
process (right path), involving the growth of an uncharged solute-sized
cavity within the bulk solvent, the transfer of the solute charge
distribution from vacuum into cavity, and the interaction between
the solute and solvent.}
\par\end{center}%
\end{minipage}
\end{figure}

As not all kinds of interactions are important in applications, different
models and methods have been developed according to the usage.

For most of the 20th century, the study of solvation effects has been
dominated by continuum (implicit) models \citep{Jensen,Cramer_1999},
which mostly rely on the continuum dielectric description of the solvent
and are not costly in terms of computation resources. They provide
an accurate way to treat the strong, long-range electrostatic interactions
which dominate many solvation phenomena, but lack detailed information
on the first solvation shell. This information mainly includes the
cavity formation energy and solute-solvent van der Waals interactions,
which are often roughly treated by introducing an artificial form
of cavity that links to the form of the solute. The methods for testing
electrostatic interactions include the Generalized Born (GB) approximation
or, for better estimates, Poisson-Boltzmann (PB) calculations. These
are widely integrated within \acs{QM} calculations by adding extra
solvation terms onto the Fock or Kohn-Sham operator \citep{Tomasi_1994_implicit_model,tomasi_quantum_2005}.
However, the improper treatment of the first shell, where the microscopic
interactions are primarily located, often introduces potentially huge
errors in free energy evaluation, especially for polar solvents (such
as water), despite the accuracy that the \acs{QM} calculation alone
can achieve. Therefore, classical molecular simulations, which describe
the individual solvent molecules explicitly (explicit solvent), particularly
the molecular dynamics (\acs{MD}) and Monte Carlo method (\acs{MC}),
have become the alternative solution during the last few decades.
They generate trajectories and configurations, and from there estimate
free energy changes by statistical mechanical techniques, such as
free energy perturbation (FEP) theory or thermodynamic integration
(TI) \citep{Jorgensen_1995_MC}. These calculations are very demanding
in computing cost, due to the need for numerous (hundreds or thousands)
solvent molecules to form a realistic model, and an even greater number
of configurations (millions) in order to be statistically significant.

Recently, a third domain of theory to describe solvents based on the
statistical mechanics of fluids has been growing rapidly. It mainly
involves the integral equation theory (\acs{IET}), and the classical
density functional theory (c\acs{DFT}) for liquids. These approaches
are capable of giving the molecular nature of the first solvation
shell, not by calculating all the instantaneous micro-states with
respect to time, but rather by theoretically integrating over positions
and momenta. Therefore, they are orders of magnitude faster than the
simulations done by micro-states.

The integral equation theory focuses on solving the Ornstein-Zernike
(\acs{OZ}) equation with a specific closure equation \citep{Hensen-McDonald,Gray-Gubbins}.
It was initially limited to so-called ``simple liquids'' - a system
of spherical particles. The extension to molecular fluids, composed
of polyatomic molecules with non-spherical shapes, was done in two
different directions. On the one hand, Chandler and Andersen in 1971
\citep{Chandler_1972_RISM} developed the reference interaction site
model (RISM), which discretizes the distribution and correlation functions
into site-site functions, and solves somewhat phenomenological \acs{OZ}
and closure equations in a matrix form \citep{hirata_molecular_2004}.
On the other hand, Blum \citep{Blum_I,Blum_II,blum_III}, Fries and
Patey \citep{Fries_Patey_1985} extend the \acs{OZ} equation into
a full molecular form, where the distribution and correlation functions
depend on both position and orientation. In their theory, the orientation
part of \acs{OZ} equation is simplified by expanding the distribution
and correlation functions onto rotational invariants, which can be
expressed in terms of Wigner generalized spherical harmonics.

The classical density functional theory approach deals with inhomogeneous
liquids, and uses the same variation principle and minimization strategy
\citep{mermin_thermal_1965,Evans_1979,Hansen_1987} as electronic
density functional theory (e\acs{DFT}) for electron-electron interactions.
The latter has received immense success in computational chemistry.
Classical \acs{DFT} gives the solvation grand potential (usually
named as free energy) and the equilibrium solvent density by minimizing
the free energy functional of the solvent density in the presence
of a given external potential. Borgis and collaborators \citep{gendre_classical_2009,jeanmairet_molecular_2013-1,jeanmairet_molecular_2015,jeanmairet_molecular_2016,Jeanmairet_thesis,levesque_solvation_2012,ramirez_density_2002,ramirez_density_2005,sergiievskyi_fast_2014,Zhao_2011}
have recently generalized it into the molecular case, leading to molecular
density functional theory (\acs{MDFT}), where the solvent density
depends on both position and orientation, $\rho(\mathbf{r},\mathbf{\Omega})$.
The main theoretical difficulty lies in the definition of well-funded
and reliable functionals of the excess free energy $\mathcal{F}_{\mathrm{exc}}\left[\rho\right]$,
accounting for the geometric complexity of the solvent molecule. Some
recent research has shown that \acs{MDFT} is capable of describing
linear solvents like acetonitrile, but still has some caveats for
the most complex solvent, water \citep{Zhao_2011}. \acs{MDFT} can
be proven as mathematically equivalent to the two-component molecular
\acs{IET}, in the limit that the functional is continuous (grid infinitely
fine) and in an infinite system.

The majority of work of all these theories has been focused on water,
since it is one of the most difficult systems to model due to its
molecular geometry, unavoidable multi-body character, quantum effects,
and hydrogen bonds, to name a few. The importance of including instantaneous
polarization in potential functions is also an issue \citep{polarisable_1,polarisable_2}.
However, since polarizable force fields are not yet in common use,
the simulations by micro-states and the liquid theory which feed on
force fields also have their own limits, compared to the continuum
model which can be totally polarizable. The advantages and disadvantages
of each branch of theory are listed in table \ref{tab:Theories-of-solvation}.

\begin{table}[h]
\begin{centering}
\begin{tabular}{ccccc}
\toprule 
\tableheadline{Theory} & \tableheadline{Speed} & \tableheadline{Long-Range} & \tableheadline{First-Shell} & \tableheadline{Polarizable Solvent}\tabularnewline
\midrule
Continuum model & fast & yes & no & fully\tabularnewline
Simulation by time & costly & yes & yes & partially, very costly\tabularnewline
Theory of liquids & fast & yes & yes & partially\tabularnewline
\bottomrule
\end{tabular}
\par\end{centering}
\caption{Solvation theories\label{tab:Theories-of-solvation}}
\end{table}


\section{Scope of this thesis}

This thesis aims at developing the theory and the code of \acs{MDFT},
focusing on the generalization and algorithmic acceleration of the
excess free energy functional $\mathcal{F}_{\mathrm{exc}}$ evaluation
under homogenous reference fluid (\acs{HRF}) approximation, which
will be discussed in detail in later chapters. 

Chapter I reviews a selection of models and methods to describe solvent
effects. It includes the implicit and explicit models, the basics
of liquid-state theory, as well as its two frontier research domains,
\acs{MDFT} and \acs{IET}. Some details of the code \acs{MDFT},
associated to the \acs{MDFT} approach, on which all the developments
of this thesis are based, are also included.

Chapter II presents all the theory developed and newly used in this
thesis. Two algorithms for the excess energy functional evaluation
under \acs{HRF} approximation are proposed. One is an extension of
the previous algorithm which could be applied to only linear solvents
(or linearized molecular solvents), to a full 3D molecular solvent
case; while the other is a new algorithm that integrates the molecular
\acs{OZ} equation treatment of angular convolution into \acs{MDFT}.
The solvation properties that the code generates are also presented,
mainly containing the corrections of free energy and solvent structure
profiles.

Chapter III reports all the implementation results, which are divided
into two aspects: the ``accuracy'', which involves the error evaluations,
comparisons between algorithms, and with \acs{IET} results; and the
``efficiency'', which evaluates the computing cost, from the parts
of the code to the entire branches.

Chapter IV gives applications to some LJ centers, ions and small molecules.
Some works that remain unachieved are put in the perspectives.
