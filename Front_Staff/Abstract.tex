\begingroup
\let\clearpage\relax
\let\cleardoublepage\relax

\pdfbookmark[1]{Abstract}{Abstract}

\chapter*{Abstract}

Solvation properties play an important role in chemical and bio-chemical
issues. The molecular density functional theory (\acs{MDFT}) is one
of the frontier numerical methods to evaluate these properties, in
which the solvation free energy functional is minimized for an arbitrary
solute in a periodic cubic solvent box. In this thesis, we work on
the evaluation of the excess term of the free energy functional under
the homogeneous reference fluid (\acs{HRF}) approximation, which
is equivalent to hypernetted-chain (\acs{HNC}) approximation in integral
equation theory. Two algorithms are proposed: the first one is an
extension of a previously implemented algorithm, which makes it possible
to handle full 3D molecular solvent (depending on three Euler angles)
instead of linear solvent (depending on two angles); the other one
is a new algorithm that integrates the molecular Ornstein-Zernike
(\acs{OZ}) equation treatment of angular convolution into \acs{MDFT},
which in fact expands the solvent density and the functional gradient
on generalized spherical harmonics (\acs{GSH}s). It is shown that
the new algorithm is much more rapid than the previous one. Both algorithms
are suitable for arbitrary three-dimensional solute in liquid water,
and are able to predict the solvation free energy and structure of
ions and molecules.

\vfill{}

\pdfbookmark[1]{Zusammenfassung}{Zusammenfassung}

\chapter*{Résumé}

\selectlanguage{french}%
Les propriétés de solvatation jouent un rôle important dans les problèmes
chimiques et biochimiques. La théorie fonctionnelle de la densité
moléculaire (\foreignlanguage{american}{\acs{MDFT}}) est l'une des
méthodes frontières pour évaluer ces propriétés, dans laquelle une
fonction d'énergie libre de solvatation est minimisée pour un soluté
arbitraire dans une boîte de solvant cubique périodique. Dans cette
thèse, nous travaillons sur l'évaluation du terme d'excès de la fonctionnelle
d’énergie libre sous l’approximation du fluide de référence homogène
(\foreignlanguage{american}{\acs{HRF}}), équivalent à l'approximation
de la chaîne hypernettée (\foreignlanguage{american}{\acs{HNC}})
dans la théorie des équations intégrales. Deux algorithmes sont proposés:
le premier est une extension d'un algorithme précédent, qui permet
de traiter le cas d'un solvant moléculaire à trois dimensions (en
fonction de trois angles d'Euler) au lieu d'un solvant linéaire (selon
deux angles); L'autre est un nouvel algorithme qui intègre le traitement
de la convolution angulaire de l'équation Ornstein-Zernike (\foreignlanguage{american}{\acs{OZ}})
moléculaire dans \foreignlanguage{american}{\acs{MDFT}}, et en fait
développe la densité du solvant et le gradient fonctionnel en harmoniques
sphériques généralisées (\foreignlanguage{american}{\acs{GSH}}s).
On montre que le nouvel algorithme est beaucoup plus rapide que le
précédent. Les deux algorithmes sont appropriés pour des solutés arbitraires
tridimensionnel dans l'eau liquide, et pour prédire l'énergie libre
et la structure de solvatation d'ions et de molécules.

\selectlanguage{american}%
\vfill{}

\endgroup
