\begingroup
\let\clearpage\relax
\let\cleardoublepage\relax

\pdfbookmark[1]{Abstract}{Abstract}

\chapter*{Abstract}

Solvation properties often play an important role in chemical and
bio-chemical issues. The molecular density functional theory (\acs{MDFT})
is one of the frontier domains to evaluate these properties, in which
the free energy functional is minimized for an arbitrary solute in
a periodic cubic solvent box. In this thesis, we work on the evaluation
of the excess term of the free energy functional under the homogeneous
reference fluid (\acs{HRF}) approximation. Two algorithms are proposed:
the first one is an extension of the previous algorithm, which allows
to calculate full 3D molecular solvent (depending on three Euler angles)
instead of linear solvent (depending on two angles); the other one
is a new algorithm that integrates the molecular \acs{OZ} equation
treatment of angular convolution into \acs{MDFT}, which in fact expands
the solvent density and the functional gradient on generalized spherical
harmonics (\acs{GSH}s). It is shown that the new algorithm is much
more rapid than the previous one, while the latter is more stable
in terms of convergency, specially for negative charged solutes. Both
algorithms are successful to predict free energy and structure of
ions and small molecules. 

\vfill{}

\pdfbookmark[1]{Zusammenfassung}{Zusammenfassung} 

\chapter*{Résumé}

\vfill{}

\endgroup
